\begin{resumen}%
En este trabajo se implementó un método alternativo al \textit{Kernel Density Estimation} (KDE) para la generación de fuentes distribucionales en simulaciones Monte Carlo, empleando histogramas multidimensionales dentro del entorno de los códigos abiertos \texttt{KDSource} y \texttt{OpenMC}. La metodología propuesta implica el procesamiento jerárquico utilizando histogramas multidimensionales para estimar distribuciones de partículas, garantizando la preservación de correlaciones en el espacio de fases.

Se desarrolló un esquema de \textit{bineado} adaptativo, capaz de captar regiones con cambios bruscos en las distribuciones, logrando así mejorar la resolución local del histograma. La validación del método se llevó a cabo mediante la comparación entre simulaciones y datos experimentales obtenidos previamente por el Departamento de Neutrones del Centro Atómico Bariloche, en el conducto N°5 del reactor RA-6. Los resultados mostraron una adecuada reproducción de los datos experimentales, verificando la efectividad del método implementado.

\end{resumen}

\begin{abstract}%
In this work, an alternative method to \textit{Kernel Density Estimation} (KDE) was implemented for generating distributional sources in Monte Carlo simulations, employing multidimensional histograms within the open-source code frameworks \texttt{KDSource} and \texttt{OpenMC} frameworks. The proposed methodology involves hierarchical processing using multidimensional histograms to estimate particle distributions, preserving the correlations among phase space.

An adaptive binning scheme was developed to effectively capture regions with abrupt changes in the distributions, thereby enhancing local histogram resolution. Validation of the method was performed by comparing simulation results with experimental data previously obtained by the Neutron Department of \textit{Centro Atómico Bariloche}, in beam N°5 of the RA-6 research reactor. Results demonstrated good agreement with the experimental data, confirming the effectiveness of the implemented method.

\end{abstract}



