\chapter{Evaluación del Método mediante Resampleo en \emph{Trackfiles}}
\label{chap:evaluacion-trackfiles}

\paragraph{Configuración 2: binning adaptativo (adaptive/adaptive).}  
Al aplicar binning adaptativo tanto en los macrogrupos como microgrupos, se optimiza la ubicación de los bordes para reflejar mejor la densidad local. En letargia se observa una mejora sustancial en la representación de los valores tipo delta, aunque persisten irregularidades en las zonas de letargia alta. Las variables espaciales muestran una mayor suavidad local sin perder detalle global.

Las métricas KL mejoran drásticamente respecto del caso uniforme:

\begin{center}
\begin{tabular}{lcc}
\toprule
\textbf{Configuración} & $\sum$KL 1D [nats] & $\sum$KL 2D [nats] \\
\midrule
Adaptive / Adaptive & 0.533 & 10.15 \\
\bottomrule
\end{tabular}
\end{center}

\paragraph{Configuración 3: macrobinning adaptativo, microbinning uniforme.}  
Esta variante mejora parcialmente la resolución estructural de la distribución, especialmente en 2D, pero mantiene el error en 1D cuando hay discontinuidades abruptas. Esto se refleja en un KL total intermedio:

\begin{center}
\begin{tabular}{lcc}
\toprule
\textbf{Configuración} & $\sum$KL 1D [nats] & $\sum$KL 2D [nats] \\
\midrule
Adaptive / Equal & 1.087 & 12.56 \\
\bottomrule
\end{tabular}
\end{center}

\paragraph{Configuración 4: macrobinning uniforme, microbinning adaptativo.}  
Esta última configuración aún no ha sido evaluada en detalle, pero se espera que muestre un comportamiento intermedio entre los casos previos. Se incluirá su análisis en una versión posterior.



\section{Resultados con Trackfile 1}
\label{sec:resultados-track1}

En esta sección se presentan los resultados obtenidos al aplicar el método sobre el primer archivo de tracks. El análisis se realizó utilizando múltiples configuraciones, variando el orden de las variables, la cantidad de macrogrupos y microgrupos.

Se muestran:

\begin{itemize}
    \item Distribuciones de energía, posición y dirección reconstruidas y su comparación con las originales.
    \item Curvas de error relativo por variable.
    \item Mapas de error relativo 2D para las correlaciones seleccionadas.
    \item Tabla con valores de divergencia KL para distintas configuraciones.
\end{itemize}

Estos resultados permiten discutir la sensibilidad del método a cada uno de los parámetros de entrada y establecer lineamientos para su elección óptima.

\subsection{Comparación entre esquema adaptativo y de ancho constante}
\label{subsec:adaptativo-vs-constante}

Como parte del análisis sobre el primer trackfile, se estudió el impacto del esquema de discretización utilizado en los macro y microgrupos. Se comparó explícitamente el rendimiento del esquema adaptativo frente a un esquema de histogramas con ancho constante.

El esquema adaptativo asigna mayor resolución a regiones del espacio de fases con alta estadística, y menor resolución donde los datos son escasos. Esto permite representar con mayor fidelidad las distribuciones sin amplificar el ruido estadístico.

Los resultados muestran una mejora significativa en la reconstrucción cuando se utiliza el esquema adaptativo. Esto se evidencia tanto en los gráficos de error relativo como en los valores de la divergencia KL, donde se observa una reducción consistente al aplicar el bineado adaptativo.

Esta comparación resalta la importancia de emplear técnicas de bineado adaptativo como herramienta para preservar la calidad del muestreo, especialmente en regiones con estructuras finas y alta variabilidad estadística.

\section{Resultados con Trackfile 2}
\label{sec:resultados-track2}

Se repitió el procedimiento metodológico sobre un segundo archivo de tracks con características geométricas y espectrales diferentes al primero, lo cual permite poner a prueba la generalidad del método propuesto.

Al igual que en el caso anterior, se analizaron distintas configuraciones de discretización jerárquica y orden de variables, prestando especial atención a:

\begin{itemize}
    \item La estabilidad del método ante distribuciones menos suaves o más concentradas espacialmente.
    \item La sensibilidad de las correlaciones bidimensionales al cambio en el esquema de macrogrupos.
    \item El comportamiento de la divergencia KL como función de la resolución utilizada.
\end{itemize}

Los gráficos comparativos muestran una buena reconstrucción general de las distribuciones 1D, aunque se observaron mayores errores relativos en regiones de baja estadística. En cuanto a las correlaciones, se destaca nuevamente la importancia de evitar configuraciones con fragmentación excesiva en las últimas variables del árbol.

Se presenta a continuación una selección representativa de los resultados gráficos y numéricos obtenidos, incluyendo:

\begin{itemize}
    \item Plots de distribuciones originales vs. reconstruidas.
    \item Mapas de error relativo 2D para variables espaciales y angulares.
    \item Tabla comparativa de divergencias KL.
\end{itemize}

\begin{table}[H]
    \centering
    \caption{Divergencia KL para distintas configuraciones en Trackfile 2.}
    \begin{tabular}{lccc}
        \toprule
        \textbf{Configuración} & \textbf{Macrogrupos} & \textbf{Microgrupos} & \textbf{KL Divergence} \\
        \midrule
        Orden A & [8,8,8,8] & [80,80,80,80] & 0.021 \\
        Orden B & [6,7,8,9] & [60,70,80,90] & 0.016 \\
        Orden C & [9,8,7,6] & [100,80,60,40] & 0.019 \\
        \bottomrule
    \end{tabular}
    \label{tab:trackfile2_kl}
\end{table}


\section{Resultados con Trackfile 3}
\label{sec:resultados-track3}

Finalmente, se aplicó el método sobre un tercer archivo de tracks con características mixtas, presentando una distribución energética más extendida y un patrón angular complejo, lo cual representa un desafío adicional para el muestreo jerárquico.

Se realizaron pruebas similares a las anteriores, enfocándose en:

\begin{itemize}
    \item Evaluar la robustez del método frente a distribuciones con múltiples picos o simetrías rotas.
    \item Observar el efecto del esquema de macrogrupos decreciente en variables angulares.
    \item Validar si se mantiene la tendencia en la divergencia KL al aplicar bineado adaptativo.
\end{itemize}

Los resultados obtenidos son consistentes con los de los otros casos, aunque se observaron diferencias notables en la reconstrucción de las variables angulares cuando éstas se encontraban en posiciones tempranas dentro del árbol, lo cual sugiere una posible pérdida de fidelidad debido a fragmentación excesiva.

\begin{table}[H]
    \centering
    \caption{Divergencia KL en Trackfile 3 para diferentes combinaciones.}
    \begin{tabular}{lccc}
        \toprule
        \textbf{Configuración} & \textbf{Macrogrupos} & \textbf{Microgrupos} & \textbf{KL Divergence} \\
        \midrule
        Esquema uniforme & [6,6,6,6] & [60,60,60,60] & 0.024 \\
        Esquema creciente & [6,7,8,9] & [60,70,80,90] & 0.017 \\
        Esquema adaptativo & Variable & Adaptativo & 0.012 \\
        \bottomrule
    \end{tabular}
    \label{tab:trackfile3_kl}
\end{table}

Estos resultados reafirman la utilidad del esquema adaptativo en escenarios complejos, donde las estructuras locales en el espacio de fases requieren una resolución flexible para ser correctamente representadas.


