%%%%%%%%%%%%%%%%%%%%%%%%%%%%%%%%%%%%%%%%%%%%%%%%%%%%%%%%%%%%%%%%%%%%%%%%%%%%%%%%
% \documentclass[12pt,papel,twoside]{ibtesis}
% \documentclass[12pt,papel,singlespace,oneside]{ibtesis}
% \documentclass[12pt,papel,preprint,singlespace,oneside]{ibtesis}

% \documentclass[screen,pagebackref]{ibtesis}
% Antes acá estaba
\documentclass[12pt,screen,twoside,pagebackref]{ibtesis}


%%%%%%%%%%%%%%%%%%%%% Paquetes extra %%%%%%%%%%%%%%%%%%%%%%%%%%%%%%%%%%%%%%%%%%%
% Por conveniencia: aqu\'{\i} puede cargar todos los paquetes y definir los comandos 
% que necesite
\usepackage{ibextra}
\usepackage{booktabs}
\usepackage{float}
\usepackage{subcaption}
% \usepackage{subfigure}
\usepackage[utf8]{inputenc}

%\usepackage{hyphen-spanis}
%%%%%%%%%%%%%%%%%%%%%%%%%%%%%%%%%%%%%%%%%%%%%%%%%%%%%%%%%%%%%%%%%%%%%%%%%%%%%%%%
%%%%%%%%%%%%%%%%%%%%% Informacion sobre la tesis %%%%%%%%%%%%%%%%%%%%%%%%%%%%%%%
\title{Incorporación de técnicas de muestreo mediante histogramas multidimensionales al código de
simulación de fuentes de Monte Carlo KDSource}
\author{Lucas Ezequiel Ovando}
\director{Dr. Ariel Marquez}
\codirector{Ing. Zoe Prieto}
\carrera{Proyecto Integrador de la Carrera de Ingeniería Nuclear}
\grado{Estudiante}
\laboratorio{Departamento de Fisica de Reactores y Radiaciones}
\jurado{Dr. Edmundo Lopasso \\ 
Mg. Norberto Schmidt}
\palabrasclave{Monte Carlo, Histogramas multidimensionales, OpenMC, KDSource, RA-6, Instituto Balseiro}
\keywords{Monte Carlo, Multidimensional histograms, OpenMC, KDSource, RA-6, Instituto Balseiro}
% Si queremos poner la fecha manualmente:
% \date{Diciembre de 2099}

%%%%%%%%%%%%%%%%%%%%%%%%%%%%%%%%%%%%%%%%%%%%%%%%%%%%%%%%%%%%%%%%%%%%%%%%%%%%%%%%
%\titlepagefalse % Si no quiere compilar la portada descomente esta linea
%\includeonly{apendices} % Compilar s\'{o}lo estos archivos 
\graphicspath{{figs/}} % Lugar donde encontrar las figuras generales (se puede poner uno en cada cap{\'{\i}}tulo)
%%%%%%%%%%%%%%%%%%%%%%%%%%%%%%%%%%%%%%%%%%%%%%%%%%%%%%%%%%%%%%%%%%%%%%%%%%%%%%%%


\begin{document}

% Dentro del environment 'preliminary' va:
% la dedicatoria, resumen, abstract, indices

\begin{preliminary}

% Escriba su dedicatoria
\dedicatoria{
A todos por igual
}

%%% \'{I}ndices %%%%

\begin{abreviaturas}
\begin{itemize}
    \item API: Application Programming Interface
    \item RAM: Random Access Memory.
    \item KDE: Kernel Density Estimation.
    \item OpenMC: código Monte Carlo de transporte de partículas.
    \item KDSource: herramienta para generar fuentes distribucionales para simulaciones Monte Carlo.
    \item MCPL: formato de listas de partículas (Monte Carlo Particle List).
    \item XML: formato de archivo utilizado para definir fuentes en OpenMC/KDSource.
    \item On-the-fly: muestreo realizado dinámicamente durante la ejecución de una simulación.
    \item Bin: intervalo en el que se discretiza una variable para construir un histograma.
    \item Espacio de fases: conjunto de variables que definen el estado de una partícula.
\end{itemize}                            
\end{abreviaturas}

\tableofcontents                %\'{I}ndice

\listoffigures                  %Figuras

\listoftables                   %Tablas

\begin{resumen}%
En este trabajo se implementó un método alternativo al \textit{Kernel Density Estimation} (KDE) para la generación de fuentes distribucionales en simulaciones Monte Carlo, empleando histogramas multidimensionales dentro del entorno \texttt{KDSource} y \texttt{OpenMC}. La metodología propuesta implica el procesamiento jerárquico utilizando histogramas multidimensionales para estimar distribuciones de partículas, garantizando la preservación de correlaciones entre variables energéticas, espaciales y angulares.

Se desarrolló un esquema de bineado adaptativo, capaz de captar regiones con cambios bruscos y deltas en las distribuciones, logrando así mejorar la resolución local del histograma. La validación del método se llevó a cabo mediante la comparación entre simulaciones y datos experimentales obtenidos previamente por el Departamento de Neutrones del Centro Atomico Bariloche, en el conducto N°5 del reactor RA-6. Los resultados mostraron una adecuada reproducción de los datos experimentales, verificando la efectividad del método implementado.

\end{resumen}

\begin{abstract}%
In this work, an alternative method to \textit{Kernel Density Estimation} (KDE) was implemented for generating distributional sources in Monte Carlo simulations, employing multidimensional histograms within the \texttt{KDSource} and \texttt{OpenMC} frameworks. The proposed methodology involves hierarchical processing using multidimensional histograms to estimate particle distributions, ensuring the preservation of correlations among energy, spatial, and angular variables.

An adaptive binning scheme was developed to effectively capture regions with abrupt changes and delta-like features in the distributions, thereby enhancing local histogram resolution. Validation of the method was performed by comparing simulation results with experimental data previously obtained by the Neutron Department of Centro Atomico Bariloche, in channel N°5 of the RA-6 reactor. Results demonstrated good agreement with the experimental data, confirming the effectiveness of the implemented method.

\end{abstract}





\end{preliminary}

%% acá deberían incluirse los capítulos
\chapter{Introducción}

\section{Contexto y motivación}

Las simulaciones Monte Carlo son la herramienta estándar cuando la complejidad geométrica o la marcada anisotropía angular del flujo neutrónico dificultan la aplicación de métodos determinísticos. Sin embargo, cuando las partículas atraviesan blindajes o regiones altamente absorbentes, la estadística disponible en la zona de interés se reduce considerablemente, lo que incrementa la incertidumbre en magnitudes físicas clave como el flujo escalar o la dosis equivalente ambiental, e incluso puede imposibilitar su cálculo debido a nula estadística. 

En este trabajo abordaremos la temática de la obtención de resultados a la salida de canales de extracción de neutrones en reactores de investigación tipo pileta. En estos sistemas, los neutrones generados en el núcleo deben atravesar el agua de la pileta antes de alcanzar la entrada del canal de extracción. Durante una simulación desde el núcleo, la mayoría de los neutrones permanecen dentro del mismo y aquellos que logran salir al agua son en gran medida absorbidos a medida que penetran en esta. En consecuencia, son escasos los neutrones que finalmente alcanzan la entrada del canal, haciendo que resulte computacionalmente costoso obtener estadística suficiente a la salida del canal, en el ambiente circundante o en dispositivos experimentales.

En la Figura \ref{fig:nucleo2} se presenta un esquema ilustrativo del núcleo de un reactor de investigación tipo pileta con un canal de extracción de neutrones. Este esquema consta de un núcleo central rodeado por agua y rodeado externamente por un blindaje biológico. El canal de extracción permite obtener un haz de neutrones destinado a usos posteriores en investigaciones experimentales. Además, se muestra esquemáticamente mediante un gradiente de color las regiones con mayor estadística neutrónica y cómo esta disminuye drásticamente al avanzar por el agua y el blindaje. Este esquema ejemplifica una simulación típica de núcleo, ilustrando claramente las dificultades que se tienen para obtener estadística lejos del núcleo.

\begin{figure}[H]
    \centering
    \includegraphics[width=0.8\textwidth]{nucleo2.png}
    \caption{Representación esquemática del núcleo de un reactor tipo pileta con canal de extracción. Se observa el núcleo central, rodeado por agua y blindaje biológico, así como un canal de extracción de neutrones. El gradiente de color indica cualitativamente la distribución de la población neutrónica típica en simulaciones Monte Carlo desde el núcleo.}
    \label{fig:nucleo2}
\end{figure}

Para mitigar esta problemática, se suelen emplear técnicas de reducción de varianza tales como:
\begin{itemize}
    \item \textbf{Separación geométrica del problema}: consiste en dividir la simulación en regiones contiguas, separando la fuente original de la región de interés mediante una superficie de interfaz. En dicha superficie se registran las variables del espacio de fases y el peso estadístico de las partículas que la atraviesan, con el objetivo de construir una nueva fuente de partículas estadísticamente equivalente. Esta nueva fuente permite reiniciar la simulación a partir de dicha superficie, concentrando los recursos computacionales exclusivamente en la región de interés. Existen diversos métodos para modelar este tipo de fuentes distribucionales a partir del archivo de partículas registrado, tales como el ajuste de funciones unidimensionales o multidimensionales, el método de \textit{smearing} utilizado en \textit{McStas} \cite{McStas2020Manual}, la estimación por densidad de kernels (\textit{Kernel Density Estimation}, KDE) adaptativa \cite{Schmidt2022KDSourcePaper}, o el uso de histogramas multidimensionales \cite{Fairhurst2017Hist}.
    
    \item \textbf{Absorción implícita}: en lugar de eliminar partículas cuando son absorbidas, se les reduce su peso estadístico de acuerdo a la probabilidad de absorción, permitiendo que las partículas sobrevivan y continúen su historia.
    
    \item \textbf{Ruleta rusa}: se aplica para eliminar partículas con bajo peso estadístico, preservando el valor esperado del cálculo. Esta técnica se utiliza típicamente en conjunto con la absorción implícita, ya que sin ella todas las partículas mantienen peso unitario y la aplicación de ruleta rusa no sería necesaria. Cada partícula con peso inferior a un umbral predefinido tiene una cierta probabilidad de ser eliminada; si sobrevive, su peso es incrementado.

    \item \textbf{Ventanas de peso}: esta técnica define un rango aceptable de pesos estadísticos para las partículas en cada región del dominio. Aquellas con peso muy alto son divididas (\emph{splitting}) y las de peso muy bajo se someten a ruleta rusa para controlar la varianza espacialmente.
\end{itemize}

El método desarrollado en este trabajo se basa en la separación geométrica del problema en dos regiones: una que contiene el núcleo del reactor, y otra que abarca exclusivamente la región de interés. Esta técnica se implementa mediante la división del modelo simulado en dos regiones, delimitadas por superficies de acople ubicadas estratégicamente dentro de la geometría. En una primera etapa, se realiza una simulación completa del núcleo hasta una superficie intermedia, donde se registran en una lista las propiedades de las partículas que la atraviesan, incluyendo su energía E, posición \textbf{r}, dirección $\mathbf{\Omega}$, peso estadístico, y tipo de partícula (por ejemplo, neutrón o fotón). A partir de esta lista de partículas, se estima la distribución multidimensional en el espacio de fases utilizando histogramas multidimensionales. Esta estructura permite segmentar el conjunto original de datos mediante histogramas de baja resolución, llamados histogramas macro, que separan el espacio de fases en regiones donde las correlaciones entre variables se mantienen aproximadamente constantes. Sobre cada una de estas regiones, se construyen histogramas de mayor resolución —los histogramas micro— que aproximan la distribución de cada variable con mayor detalle. De este modo, se preservan tanto la forma general de la distribución como las correlaciones entre las variables registradas. A partir de esta estimación, es posible generar nuevas partículas que pertenezcan estadísticamente al mismo espacio de fases que la muestra original, obteniendo así una fuente sintética denominada fuente distribucional. Esta metodología se desarrolla en profundidad en el Capítulo \ref{cap:metodo_histogramas}, donde se describe el procedimiento implementado para su construcción.

En la Figura~\ref{fig:nucleo4} se ejemplifica el proceso desarrollado para una simulación del núcleo de un reactor de investigación tipo pileta. Inicialmente, se realiza una simulación completa desde el núcleo, obteniendo una estadística adecuada en sus alrededores, la cual se reduce considerablemente a medida que los neutrones penetran en el agua circundante. Este fenómeno se visualiza mediante un gradiente de colores. En esta simulación se define una superficie de registro en la entrada de un canal de extracción, y las partículas que la atraviesan se registran en un archivo. A partir de dicho archivo, se construye una fuente distribucional mediante la metodología desarrollada en este trabajo. Esta fuente permite simular únicamente el canal, concentrando la estadística en esa región sin necesidad de simular desde el núcleo. En esta segunda simulación, la región previa a la superficie de interfaz se reemplaza completamente por vacío, dejando en el modelo únicamente la región de interés. Esto se logra definiendo una condición de vacío en la superficie de interfaz. Cabe destacar que, si bien podría existir la posibilidad de que alguna partícula inicialmente dirigida hacia la región de la fuente reingrese a la región de interés por retrodispersión, dicho fenómeno ya está estadísticamente incorporado en la lista original de partículas registrada en la primera simulación.

En resumen, el procedimiento consta de tres etapas fundamentales:

\begin{itemize}
    \item \textbf{Detección}: Registro de las variables del espacio de fases y del peso estadístico de cada partícula que atraviesa una superficie intermedia de desacople. El resultado de este proceso es un archivo de partículas que contiene la información necesaria para caracterizar la distribución de la fuente sobre dicha superficie.

    \item \textbf{Estimación}: Procesamiento del archivo de partículas mediante la metodología propuesta en este trabajo, basada en histogramas multidimensionales. Este procedimiento permite aproximar la distribución en el espacio de fases, preservando las correlaciones entre variables, y genera como resultado una fuente distribucional.

    \item \textbf{Producción}: Utilización de la fuente distribucional estimada para generar nuevas partículas que pertenezcan al espacio de fases del archivo de partículas original. Estas partículas son empleadas en simulaciones posteriores, permitiendo modelar la región de interés de forma desacoplada de la fuente original.
\end{itemize}

\begin{figure}[H]
    \centering
    \includegraphics[width=\textwidth]{nucleo4.png}
    \caption{Esquema ilustrativo del proceso de desacople geométrico en simulaciones Monte Carlo para un reactor de investigación tipo pileta. En la primera etapa se registra un archivo de partículas en la entrada del canal de extracción, el cual es posteriormente utilizado para generar una fuente distribucional que permite simular eficientemente el canal de forma independiente. El gradiente de colores representa la disminución de la población neutrónica.}
    \label{fig:nucleo4}
\end{figure}

El archivo original de partículas contiene inevitablemente regiones del espacio de fases con baja estadística o sin eventos registrados, producto de su carácter finito. Si se reutiliza directamente como fuente en una simulación posterior —por ejemplo, simulando varias veces las mismas partículas registradas— se corre el riesgo de reproducir el mismo ruido estadístico, afectando la calidad de los resultados.

% Para mitigar este problema, se emplea un procedimiento de estimación basado en histogramas multidimensionales que permite densificar el espacio de fases. En este enfoque, se utilizan histogramas para aproximar la distribución de variables, bajo el supuesto de que la probabilidad es constante dentro de cada bin. Esto habilita la generación de nuevas partículas en regiones que no estaban presentes explícitamente en el archivo original, completando así los vacíos estadísticos y asegurando una cobertura más uniforme del espacio de fases.

\section{Trabajos relacionados}

Los trabajos previos realizados en el Departamento de Física de Reactores y Radiaciones (DeFRRa) del Centro Atómico Bariloche (CAB) \cite{DeptoReactoresNeutronesCAB2025} abordaron el modelado de fuentes distribucionales mediante el uso de histogramas multidimensionales \cite{Fairhurst2017Hist, Ayala2019Hist, Abbate2020Hist}. Esta estrategia permitió capturar correlaciones entre energía, posición y dirección en geometrías planas rectangulares, siendo aplicada satisfactoriamente en estudios vinculados al reactor RA‑10.

En continuidad con dichos desarrollos, la herramienta \texttt{KDSource} — desarrollada en conjunto por la Comisión Nacional de Energía Atómica (CNEA) \cite{CNEA2025} y por el Instituto Balseiro (IB) \cite{IB2025}— introdujo técnicas más avanzadas basadas en \textit{Kernel Density Estimation} (\textit{KDE}), en forma multivariable y adaptativa \cite{Abbate2021KDSource, Schmidt2022KDSourcePaper, Fox2022KDE, Gimenez2024KDSourceOpenMC}. \texttt{KDSource} permite ajustar automáticamente distribuciones continuas a partir de archivos de partículas previamente obtenidos y, a partir de ellas generar fuentes distribucionales, preservando las correlaciones del espacio de fases. Estas fuentes pueden ser utilizadas directamente en simulaciones Monte Carlo mediante el código \texttt{OpenMC} \cite{OpenMC2024}.

No obstante, el enfoque basado en \textit{KDE} presenta una limitación: su carácter inherentemente suavizante puede dificultar la representación precisa de discontinuidades abruptas en el espacio de fases. En particular, el método puede fallar al capturar regiones donde las distribuciones presentan derivadas segundas muy marcadas o donde ocurren cambios bruscos, como bordes definidos por condiciones geométricas o materiales. En estos casos, la suavizacion de las distribuciones multivariables puede degradar la calidad de la fuente generada, especialmente cuando se busca preservar estructuras finas o anisotropías marcadas.

El presente trabajo extiende estas líneas incorporando los enfoques basados en histogramas multidimensionales dentro del entorno ya establecido de \texttt{KDSource}, proponiendo además un método de discretización adaptativa que automatiza la definición de las grillas de los histogramas macro y micro. A diferencia de los trabajos anteriores, este nuevo enfoque permite segmentaciones variables para diferentes subconjuntos del espacio de fases. Por ejemplo, el método puede asignar una discretización angular más refinada para neutrones rápidos, donde suelen prevalecer anisotropías asociadas a haces colimados, y una discretización más homogénea para neutrones térmicos, cuya distribución tiende a ser más isotrópica. De esta forma, se logra preservar las correlaciones relevantes sin imponer una malla uniforme en todo el dominio, mejorando la capacidad de representación del método.

\section{Aportes específicos de este trabajo}

Este trabajo busca profundizar el desarrollo de \texttt{KDSource} mediante la incorporación de histogramas multidimensionales como una alternativa -o complemento- a la metodología \textit{KDE} existente. Las contribuciones específicas son:

\begin{itemize}
    \item Implementación de histogramas multidimensionales en \texttt{KDSource}, capaces de:
    \begin{itemize}
        \item preservar las correlaciones entre variables ($\mathbf{E}$–$\mathbf{r}$–$\boldsymbol{\Omega}$) para fuentes definidas sobre un plano;
        \item representar adecuadamente discontinuidades o picos en todas las variables;
        \item implementar un método de selección automática y adaptativa de bordes para los histogramas, tanto a nivel macro como micro, optimizado según la estadística disponible en cada subgrupo del espacio de fases.
    \end{itemize}

    \item Integración optimizada del flujo de trabajo en \texttt{OpenMC} mediante:
    \begin{itemize}
        \item generación \emph{offline} de un archivo intermedio en formato \texttt{XML} conteniendo la fuente distribucional, que incluye la información de los histogramas multidimensionales;
        \item desarrollo de un módulo en \texttt{C} que utilice eficientemente esta información para producir partículas individualmente;
        \item implementación de un remuestreo \emph{on-the-fly} integrado en \texttt{OpenMC}, minimizando la ocupación de memoria al evitar la carga y gestión de archivos extensos de partículas.
    \end{itemize}

    \item Validación  del método en casos de complejidad creciente:
    \begin{itemize}
        \item Un caso simplificado, consistente en un haz colimado ingresando en un paralelepípedo de agua atravesado por un canal de vacío, con fuentes definidas artificialmente para permitir una comparación con una simulación directa más extensa sin la aplicación del método desarrollado. 
        \item Un caso real, consistente en la propagación a través del conducto Nº5 del reactor RA‑6, utilizando un archivo de partículas proporcionado por el departamento de Física de Neutrones del CAB, que fue generado a través de una simulación del núcleo en \texttt{OpenMC}.
    \end{itemize}
\end{itemize}

\chapter{Herramientas utilizadas}
\label{cap:herramientas}

Este capítulo tiene por objetivo presentar y describir las herramientas computacionales utilizadas a lo largo de este trabajo. Específicamente, se abordará el programa \texttt{OpenMC}, los formatos de archivos usados (\texttt{MCPL} y \texttt{XML}), y los lenguajes de programación involucrados (\texttt{Python}, \texttt{C} y \texttt{C++}).

\section{\texttt{OpenMC}}

\texttt{OpenMC} es un programa de simulación Monte Carlo desarrollado inicialmente en el \textit{Massachusetts Institute of Technology} como \textit{software} de código abierto \cite{OpenMC2024}. Actualmente es mantenido por el \textit{Argonne National Laboratory} junto a una activa comunidad internacional que contribuye continuamente a su desarrollo y expansión. Está especialmente orientado al cálculo del transporte de neutrones y fotones, permitiendo tanto simulaciones de criticidad como simulaciones de fuente fija. En este trabajo se emplean exclusivamente simulaciones de fuente fija.

Una de las ventajas de \texttt{OpenMC} es su flexibilidad en la obtención de resultados. El código permite registrar diferentes magnitudes físicas como flujos, dosis, corrientes, espectros energéticos, etc., además de ofrecer la posibilidad de registrar partículas que atraviesan superficies definidas por quien utiliza el programa en un archivo de formato \texttt{HDF5} \cite{HDF5_2025} o \texttt{MCPL} \cite{MCPL2024}. Estos archivos almacenan información de las variables de las partículas, permitiendo su posterior análisis o reutilización para generar nuevas simulaciones. Asimismo, \texttt{OpenMC} incorpora técnicas avanzadas de reducción de varianza, como las ventanas de peso (\textit{weight windows}), que resultan fundamentales para reducir la incertidumbre estadística de los resultados en áreas de interés específico, técnica utilizada en este trabajo.

\section{\texttt{KDSource}}



\texttt{KDSource} es una herramienta computacional desarrollada en conjunto por CNEA e IB, cuyo objetivo principal es procesar archivos de partículas generados en simulaciones Monte Carlo para la construcción de nuevas fuentes distribucionales \cite{KDSource2024}. Su integración con \texttt{OpenMC} permite desacoplar geométricamente simulaciones complejas, facilitando significativamente el cálculo de transporte en geometrías difíciles o extensas.

El fundamento original de \texttt{KDSource} reside en la técnica \textbf{Kernel Density Estimation (KDE)}, una técnica estadística que permite estimar distribuciones continuas de variables a partir de muestras discretas, manteniendo la correlación existente entre ellas. 

\section{Formatos de listas de partículas: \texttt{MCPL} y \texttt{HDF5}}

En simulaciones Monte Carlo, es común registrar partículas que atraviesan una superficie o ingresan a una región de interés, generando lo que se conoce como una lista de partículas. Estos archivos permiten capturar el estado de cada partícula —incluyendo su energía, posición, dirección, peso estadístico y tipo de partícula— al momento de cruzar una superficie.

\texttt{OpenMC} emplea por defecto el formato \texttt{HDF5} para almacenar estas listas \cite{HDF5_2025}. Este formato binario jerárquico permite almacenar datos de manera eficiente, estructurada y accesible desde diversos lenguajes de programación. Sin embargo, su estructura está adaptada específicamente al ecosistema de \texttt{OpenMC}, dificultando su reutilización directa en otros códigos de transporte.

Con el objetivo de facilitar la interoperabilidad entre distintos códigos Monte Carlo, existe el formato \textbf{\texttt{MCPL} (\textit{Monte Carlo Particle List})} \cite{MCPL2024}. Este estándar de listas de partículas permite almacenar listas generadas en simulaciones de forma compacta y eficiente, preservando información esencial como la energía, posición, dirección, peso de cada partícula y tipo de partícula. A diferencia de formatos específicos como \texttt{HDF5}, \texttt{MCPL} fue concebido como una interfaz  entre diferentes entornos Monte Carlo.

La Figura \ref{fig:mcpl_support_diagram} muestra los diversos códigos que pueden producir o consumir archivos en formato \texttt{MCPL}. En este proyecto, su uso permite vincular eficientemente los archivos de partículas generados por \texttt{OpenMC} con la herramienta \texttt{KDSource}.

\section{Formato \texttt{XML}}

El formato \textbf{\texttt{XML} (\textit{Extensible Markup Language})} es un lenguaje utilizado para describir y almacenar información estructurada de manera jerárquica, basándose en una organización de datos tipo árbol \cite{XML2025}. Una de sus principales ventajas es que está diseñado para ser tanto legible por máquinas como por humanos, lo que facilita su edición, inspección y validación manual durante el desarrollo. Debido a estas características, resulta particularmente adecuado para guardar las distribuciones estimadas mediante histogramas multidimensionales. A su vez, \texttt{OpenMC} emplea archivos \texttt{XML} para almacenar la configuración completa de sus simulaciones (geometría, materiales, configuración de \textit{tallies}, etc.), por lo que la elección de este formato contribuye a una integración directa y eficiente entre los componentes desarrollados.

\begin{figure}[H]
    \centering
    \includegraphics[width=\textwidth]{figs/mcpl_support_diagram.png}
    \caption[Diagrama de soporte de \texttt{MCPL}.]{Diagrama de soporte de \texttt{MCPL}. Reproducido de \cite{MCPL2024}.}
    \label{fig:mcpl_support_diagram}
\end{figure}

\section{Lenguajes de programación: \texttt{Python}, \texttt{C} y \texttt{C++}}

A lo largo del desarrollo del proyecto, se utilizaron principalmente tres lenguajes de programación:

\begin{itemize}
    \item \textbf{\texttt{Python}}: lenguaje de alto nivel especialmente adecuado para interfaces de usuario, análisis exploratorio de datos y configuración de simulaciones debido a su simplicidad, claridad y flexibilidad. \texttt{KDSource} hace uso de \texttt{Python} para la preparación y procesamiento de datos previos al remuestreo Monte Carlo. Tanto \texttt{KDSource} como \texttt{OpenMC} disponen de una interfaz en Python.

    \item \textbf{\texttt{C}}: lenguaje de programación de bajo nivel conocido por su eficiencia computacional, velocidad y control preciso sobre la gestión de memoria. Se utilizó en este trabajo para desarrollar módulos específicos encargados del remuestreo eficiente de partículas, especialmente cuando se requieren grandes volúmenes de datos. El módulo de remuestreo de \texttt{KDSource} está escrito en este lenguaje.
    
    \item \textbf{\texttt{C++}}: lenguaje de programación que extiende a \texttt{C} con capacidades de programación orientada a objetos. Esta arquitectura permite la expansión modular del código y facilita su mantenimiento. \texttt{OpenMC} está desarrollado en este lenguaje.


\end{itemize}


El desarrollo realizado en este trabajo implementa un flujo computacional para el uso de fuentes distribucionales en simulaciones Monte Carlo. Este flujo se compone de tres etapas principales: detección, procesamiento y producción, y aprovecha las capacidades de los lenguajes \texttt{Python}, \texttt{C} y \texttt{C++}.

\textbf{Detección}: se realiza mediante una simulación inicial en \texttt{OpenMC}, en la cual se registra un archivo de partículas sobre una superficie. Este archivo puede guardarse en formato \texttt{HDF5}, que es el formato nativo de \texttt{OpenMC}, y luego ser convertido al formato \texttt{MCPL} si se desea continuar con el procesamiento. Alternativamente, si \texttt{OpenMC} fue compilado con soporte para \texttt{MCPL}, puede generarse directamente en dicho formato. El archivo resultante se almacena en disco para su uso posterior. Este procedimiento se ilustra en la Figura \ref{fig:flujo_deteccion}.

\begin{figure}[H]
    \centering
    \includegraphics[width=0.75\textwidth]{flujo_codigos-Deteccion.png}
    \caption{Etapa de detección: registro de partículas en formato \texttt{MCPL} a partir de una simulación en \texttt{OpenMC}. Bloques con esquinas redondeadas representan procesos computacionales, mientras que los bloques rectangulares indican archivos que se registran o leen desde disco.}
    \label{fig:flujo_deteccion}
\end{figure}

\textbf{Procesamiento}: esta etapa se lleva a cabo en \texttt{Python} utilizando la implementación desarrollada para procesar mediante histogramas multidimensionales. A partir del archivo \texttt{MCPL}, se construye una fuente distribucional que representa la densidad de probabilidad estimada en el espacio de fases. El resultado de esta etapa es un archivo \texttt{XML}, que contiene la parametrización de la fuente y se almacena en disco como entrada para la etapa de producción. Este archivo \texttt{XML} constituye toda la información necesaria para generar nuevas partículas en simulaciones posteriores, sin requerir el acceso al archivo original de partículas. Esta característica representa una ventaja frente al método utilizado en \texttt{KDSource}, el cual necesita mantener el archivo original disponible en memoria durante la simulación posterior. Sin embargo, ambos métodos necesitan cargar en memoria el archivo de partículas original para el procesamiento. En casos donde dicho archivo posea un volumen considerable pueden derivar en un consumo elevado de memoria \textit{RAM}. La Figura \ref{fig:flujo_procesamiento} muestra un esquema de esta etapa.

\begin{figure}[H]
    \centering
    \includegraphics[width=0.55\textwidth]{flujo_codigos-Procesamiento.png}
    \caption{Etapa de procesamiento: construcción de una fuente distribucional a partir de un archivo \texttt{MCPL}. Bloques con esquinas redondeadas representan procesos computacionales, mientras que los bloques rectangulares indican archivos que se registran o leen desde disco.}
    \label{fig:flujo_procesamiento}
\end{figure}

\textbf{Producción}: esta etapa puede abordarse de dos maneras, implementada en código \texttt{C}. En la modalidad \textit{offline}, el archivo \texttt{XML} es utilizado para generar una nueva lista de partículas en formato \texttt{MCPL}, que luego puede emplearse directamente como fuente en \texttt{OpenMC} si se cuenta con soporte para este formato. En caso contrario, puede convertirse nuevamente a \texttt{HDF5}. En la modalidad \textit{on-the-fly}, se evita por completo el uso de listas intermedias: \texttt{OpenMC} accede directamente al archivo \texttt{XML} y realiza el remuestreo dinámicamente durante la simulación, gracias a una extensión \emph{ad-hoc} incorporada en este trabajo. Ambas modalidades se ilustran en la Figura \ref{fig:flujo_produccion}.

\begin{figure}[H]
    \centering
    \includegraphics[width=\textwidth]{flujo_codigos-Remuestreo.png}
    \caption{Etapa de producción: uso del archivo \texttt{XML} para generar nuevas partículas de forma offline o on-the-fly. Bloques con esquinas redondeadas representan procesos computacionales, mientras que los bloques rectangulares indican archivos que se registran o leen desde disco.}
    \label{fig:flujo_produccion}
\end{figure}




\chapter{Metodología para la generación de fuentes Monte Carlo mediante histogramas multidimensionales}
\label{cap:metodo_histogramas}
\section{Introducción general al método}
La metodología propuesta se basa en el procesamiento de archivos de partículas generados en simulaciones Monte Carlo previas, específicamente usando \texttt{OpenMC}, para definir fuentes distribucionales para simulaciones subsiguientes.

\section{Definición del espacio de fases \texorpdfstring{$(\mathbf{E}$--$\mathbf{r}$--$\boldsymbol{\Omega})$}{(E--r--Omega)}}
Para aproximar correctamente las distribuciones y correlaciones del espacio de fases se consideraron seis variables para representar la energía, posición y dirección: tres coordenadas espaciales $(x, y, z)$, dos variables direccionales definidas en coordenadas esféricas ($\mu = \cos(\theta), \phi$) y una variable energética ($E$ o letargía $ln(E_0/E)$). En la figura \ref{fig:terna} se observa una representación de las variables del espacio de fases. En situaciones como la considerada en este trabajo, donde la fuente se registra sobre una superficie plana, es posible rotar el sistema de coordenadas de modo que dicha superficie resulte perpendicular al eje $z$. Por lo tanto la coordenada $z$ permanece constante, permitiendo representar el espacio de fases con cinco variables.

\begin{figure}[h]
    \centering
    \includegraphics[width=0.5\textwidth]{figs/terna.png}
    \caption{Representación del espacio de fases con las variables $(E, x, y, z, \theta, \phi)$, donde $E$ es la energía, $x$, $y$ y $z$ son coordenadas espaciales y $\theta$ y $\phi$ son las variables direccionales.}
    \label{fig:terna}
\end{figure}

\section{Procesamiento del archivo de partículas original}
\subsection{Preprocesamiento del archivo de partículas}
Inicialmente, las partículas provenientes de la simulación Monte Carlo original se filtran seleccionando únicamente aquellas que se propagan hacia la región de interés y se separan las variables relevantes mencionadas, además del peso estadístico (\textit{weight}). Esta selección se realiza porque, en la siguiente etapa de simulación, la geometría considera un vacío en la región donde originalmente se encontraba la superficie de registro, permitiendo que únicamente las partículas que avanzan en dirección hacia la región de interés puedan ser simuladas. Las partículas que se dirigen en sentido opuesto no son relevantes para la simulación posterior, ya que en caso de que alguna partícula inicialmente dirigida hacia atrás fuese a regresar por retrodispersión, dicho comportamiento ya está estadísticamente incorporado en la lista original de partículas, producto de la simulación completa previa. En la Tabla \ref{tab:estructura_trackfile} se presenta un ejemplo de la estructura típica de un archivo de partículas.

\begin{table}[h]
    \centering
    \begin{tabular}{ccccccc}
        \toprule
        \textbf{Partícula Nº} & \textbf{Letargía} & \textbf{x [cm]} & \textbf{y [cm]} & \textbf{$\mu$} & \textbf{$\phi$ [rad]} & \textbf{Peso estadístico} \\ 
        \midrule
        1       & 4.15 &  0.23  & -1.10 &  0.85 & 3.14 & 1.00 \\
        2       & 4.95 & -0.75  &  0.40 & 0.65 & 1.57 & 1.00 \\
        3       & 5.05 &  1.10  &  0.70 &  0.45 & 0.78 & 1.00 \\
        4       & 5.30 & -0.50  & -0.90 &  0.60 & 2.35 & 1.00 \\
        5       & 4.85 &  0.85  & -0.20 & 0.95 & 1.25 & 1.00 \\
        $\vdots$ & $\vdots$ & $\vdots$ & $\vdots$ & $\vdots$ & $\vdots$ & $\vdots$ \\[0.2cm]
        100000  & 5.00 &  0.10  &  0.55 & 0.70 & 0.95 & 1.00 \\
        \bottomrule
    \end{tabular}
    \caption{Ejemplo ilustrativo de la estructura típica de un archivo de partículas. El archivo original suele contener cientos de miles de partículas.}
    \label{tab:estructura_trackfile}
\end{table}

\subsection{Utilización de histogramas macro y micro}
La metodología desarrollada en este trabajo para aproximar distribuciones y correlaciones multidimensionales en el espacio de fases de partículas provenientes de simulaciones Monte Carlo se basa en un esquema combinado de histogramas macro y micro. Esta aproximación permite gestionar eficientemente grandes volúmenes de datos manteniendo una representación precisa de la información esencial del conjunto original de partículas.

Los histogramas macro son subdivisiones jerárquicas del espacio de fases, realizadas siguiendo un orden determinado por el usuario. Por ejemplo, un posible orden es la \textit{letargía}, seguida por las coordenadas espaciales (\textit{X}, \textit{Y}), y posteriormente por la dirección (\textit{$\mu$}, \textit{$\phi$}). Este procedimiento inicia dividiendo el conjunto original de partículas en macrogrupos según la primera variable elegida (por ejemplo, \textit{letargía}). Cada macrogrupo así generado es posteriormente subdividido en nuevos macrogrupos en función de la siguiente variable (por ejemplo, la coordenada \textit{X}), repitiéndose este proceso de manera iterativa para cada variable subsiguiente. El resultado es una estructura jerárquica en forma de árbol, donde cada rama representa un subconjunto específico del archivo original de partículas, ya que incluye las cinco variables consideradas. Cada nodo en la rama corresponde a  una variable dentro de dicho subconjunto. En la Figura \ref{fig:estructura_jerarquica} se ilustra este procedimiento de subdivisión jerárquica del espacio de fases.

\begin{figure}[h]
    \centering
    \includegraphics[width=0.8\textwidth]{grupos.png}
    \caption{Esquema ilustrativo de la estructura jerárquica de macrogrupos y microgrupos formando un árbol. Se resalta claramente la jerarquía entre variables.}
    \label{fig:estructura_jerarquica}
\end{figure}

Este esquema tiene como objetivo capturar las correlaciones entre las variables del espacio de fases. Cada nodo o macrogrupo resultante comparte similitudes en todas las variables consideradas hasta ese nivel jerárquico. Por ejemplo, un macrogrupo particular tendrá partículas con distribuciones similares en \textit{letargía}, coordenadas espaciales (\textit{X}, \textit{Y}), y dirección \textit{($\mu$)}, lo cual garantiza que la distribución restante en la última variable \textit{($\phi$)} refleje adecuadamente las correlaciones subyacentes en los datos.

Cabe destacar que, en la etapa de macrodiscretización, se utilizan histogramas con baja resolución debido al crecimiento exponencial en la cantidad total de grupos generados, dado que el número total de macrogrupos es producto del número de divisiones en cada variable. Un número excesivamente alto de divisiones podría conducir a subconjuntos con muy pocas partículas, comprometiendo la calidad estadística.

Una vez establecidos los macrogrupos, cada nodo es discretizado con histogramas micro, los cuales poseen una mayor resolución para representar detalladamente la distribución de cada variable dentro del subconjunto. Sin embargo, es crucial controlar esta resolución para evitar reproducir el ruido estadístico presente en los datos originales, especialmente cuando se cuenta con un número reducido de partículas en el subconjunto.

La combinación de histogramas macro y micro permite obtener una aproximación eficiente y precisa de la distribución multidimensional del conjunto original, conservando las correlaciones entre las variables del espacio de fases. Esta estructura de histogramas multidimensionales representa, en síntesis, la información esencial extraída del archivo original de partículas para ser utilizada en simulaciones posteriores.

\subsection{Métodos de discretización aplicados a histogramas macro y micro}
En este trabajo se desarrollaron dos métodos de discretización (bineado) aplicables tanto a los histogramas macro como a los micro. El primero es el \textit{bineado uniforme}, que consiste en dividir la distribución de la variable en estudio utilizando bines equiespaciados. Este método tiene la ventaja de ser sencillo y rápido en su implementación, pero presenta la desventaja de no poder capturar adecuadamente cambios abruptos o picos aislados en las distribuciones.

El segundo método, denominado \textit{bineado adaptativo}, consiste en realizar inicialmente un bineado uniforme con una cantidad moderada de bines especificada por el usuario. Luego, este bineado inicial se refina iterativamente agregando nuevos bines donde la distribución estimada difiere en mayor medida respecto a la distribución original. El criterio para añadir un nuevo bin se basa en la comparación de la función de distribución acumulada (FDC) del bineado actual con la FDC de la distribución original calculada con alta resolución. En cada iteración, se identifica el punto con la mayor diferencia absoluta entre ambas FDC, donde se coloca un borde de bin adicional.

\textbf{Figura:} Esquema gráfico del procedimiento de bineado adaptativo, mostrando las distribuciones estimadas con pocos bines, la distribución real y la diferencia absoluta utilizada para definir nuevos bines. \textbf{TO DO}

Este enfoque adaptativo permite asignar mayor resolución donde existe una mayor cantidad de peso estadístico, y menor resolución donde el peso es escaso, logrando así un mejor seguimiento de las zonas críticas y un adecuado suavizado en regiones con poca estadística.

Además, debido a que la cantidad de partículas generalmente disminuye al profundizar en las ramas del árbol jerárquico, puede ser necesario utilizar una resolución decreciente en las discretizaciones, tanto para los histogramas macro como para los micro, lo cual también ha sido implementado en este trabajo.

Finalmente, en aquellos casos en que el usuario conozca previamente la existencia de cambios abruptos o picos específicos en la distribución, es posible informar manualmente al método sobre la ubicación de estos fenómenos, estableciendo bordes de bin específicamente en esos puntos para mejorar la precisión de la discretización.


% \section{Aproximación de distribuciones mediante histogramas micro}

% Las distribuciones de las variables en el espacio de fases pueden representarse, de forma general, mediante histogramas. La elección del método de discretización (bineado) es crucial y condiciona la calidad de la aproximación obtenida. Los tres esquemas de bineado utilizados en este trabajo son:

% \begin{itemize}
%     \item \textbf{Bineado de igual separación}: Divide el rango completo de la variable en intervalos de ancho constante. 

%     \item \textbf{Bineado de igual integral}: Divide el rango en intervalos que contienen aproximadamente la misma cantidad de peso estadístico, generando así bines de tamaño variable. Este método logra una representación más eficiente en términos estadísticos, especialmente en regiones donde la densidad cambia abruptamente.

%     \item \textbf{Bineado adaptativo}: Divide el rango utilizando una discretización iterativa que asigna mayor resolución a las zonas de alta densidad estadística y menor resolución en regiones escasamente pobladas, con el objetivo explícito de reducir el ruido estadístico manteniendo manteniendo resolución en los cambios abruptos de la distribución.
% \end{itemize}

% En este trabajo se profundiza particularmente sobre el método de bineado adaptativo. El procedimiento propuesto se realiza en dos etapas principales:

% \begin{enumerate}
%     \item \textbf{Aproximación inicial gruesa}: Se discretiza la distribución con una cantidad reducida de bines uniformemente distribuidos, típicamente utilizando cerca del 25\% del número final previsto de bines, aunque puede ser definido por el usuario.
    
%     \item \textbf{Refinamiento iterativo}: En cada iteración se evalúa la diferencia absoluta entre la distribución acumulada estimada (FDC) aproximada mediante los bines actuales y la FDC de referencia calculada con muchos bines. Se agregan bines adicionales precisamente en las regiones donde esta diferencia es máxima, mejorando progresivamente la calidad del histograma resultante.
% \end{enumerate}

% Esta estrategia adaptativa permite capturar adecuadamente las regiones críticas de la distribución, proporcionando un balance controlado entre resolución local y suavidad global.

% \section{Aproximación de distribuciones mediante histogramas adaptativos}
% La distribución de cada variable es representada mediante histogramas con bines de ancho no uniforme, seleccionados mediante una técnica de \textit{binning} adaptativo. Este procedimiento asigna mayor resolución a regiones del espacio de fases con mayor densidad estadística y menor resolución en regiones de baja estadística, con el fin de reducir el ruido estadístico.

% El proceso adaptativo se realiza en dos etapas principales:
% \begin{enumerate}
%     \item Una aproximación gruesa con una cantidad inicial de bines uniformes (aproximadamente el 25\% del total).
%     \item Ajuste iterativo agregando nuevos bines donde la diferencia absoluta entre la distribución acumulada estimada (CFD) y la distribución acumulada real (calculada con un gran número de bines) es mayor.
% \end{enumerate}


% \section{Mantenimiento de correlaciones mediante histogramas macro}

% El algoritmo desarrollado permite preservar las correlaciones entre variables del espacio de fases a través de divisiones sucesivas del conjunto de datos. Este proceso admite tres esquemas posibles de bineado para cada variable: de igual separación, de igual integral o adaptativo. En todos los casos, la variable considerada se utiliza para dividir el conjunto de partículas actual en subgrupos o \emph{macrogrupos}, que son luego tratados recursivamente.

% Particularmente, en el caso del bineado adaptativo, el procedimiento comienza con una partición inicial de baja resolución distribuidos uniformemente y se aplica un refinamiento iterativo utilizando un criterio similar al empleado en la sección anterior: se identifican las regiones donde la separación entre la distribución acumulada empírica y la distribución acumulada de referencia es mayor, y se subdividen esas regiones para aumentar la resolución local.

% Este proceso de partición se aplica a cada variable en un orden definido por el usuario, generando una estructura en forma de árbol. En cada nodo del árbol se almacena la distribución acumulada de la variable correspondiente a ese nivel, junto con las fronteras de los macrogrupos definidos. Esta estructura permite preservar las correlaciones multidimensionales entre variables, ya que cada división se realiza condicionada a las divisiones previas.


% \section{Mantenimiento de correlaciones mediante macrogrupos}
% Para preservar la correlación entre variables, se utiliza una estrategia basada en la división secuencial del conjunto de datos en macrogrupos, comenzando con una división gruesa en cuatro subconjuntos y luego refinándola iterativamente mediante el procedimiento adaptativo hasta alcanzar la cantidad deseada de macrogrupos.

% Este proceso jerárquico y secuencial se aplica iterativamente a cada una de las variables del espacio de fases, dando lugar a una estructura de árbol donde cada nodo contiene información sobre la distribución acumulada de la variable correspondiente.

% En la figura \ref{fig:estructura_jerarquica} se muestra un esquema ilustrativo de la distribución jerárquica en macrogrupos. En este caso, la letargía constituye la primera variable que se procesa y se subdivide en distintos macrogrupos en la variable X. A continuación, cada uno de estos macrogrupos en X es nuevamente dividido en macrogrupos en la variable Y. Este procedimiento se repite para las variables subsiguientes, como Y y $\mu$. De este modo, cada nodo en la estructura representa un subconjunto específico del archivo de particulas original. Posteriormente, cada subconjunto es aproximado mediante un histograma micro, permitiendo obtener una estimación detallada de la distribución asociada a ese subconjunto.



\section{Remuestreo de partículas en simulaciones Monte Carlo subsecuentes}
El proceso de generación de partículas a partir de la información guardada en los histogramas multidimensionales implica:
\begin{enumerate}
    \item Generar un número pseudoaleatorio entre 0 y 1.
    \item Interpolar dicho número en la distribución acumulada normalizada de la variable raíz del árbol.
    \item Avanzar secuencialmente por las ramas del árbol determinando valores de las variables subsecuentes hasta obtener un conjunto completo de variables del espacio de fases.
\end{enumerate}

En la Figura \ref{fig:esquema_generacion_particulas} se ejemplifica el proceso de generación de partículas descrito previamente. En el caso particular de trabajar con cinco variables, se repetirá la generación de números pseudoaleatorios cinco veces consecutivas, obteniéndose así los cinco valores correspondientes de las variables. Cabe destacar que durante el remuestreo de la quinta variable no se generará un macrogrupo adicional, dado que esta es la última variable a muestrear y no existen más niveles en el árbol.

\begin{figure}[H]
    \centering
    \includegraphics[width=\textwidth]{esquema5.png}
    \caption{Esquema ilustrativo del proceso secuencial de generación de partículas a partir de la distribución jerárquica guardada, resaltando el muestreo sucesivo mediante números pseudoaleatorios.}
    \label{fig:esquema_generacion_particulas}
\end{figure}

% Este enfoque evita la necesidad de cargar grandes listas en memoria \textit{RAM} durante la ejecución de simulaciones Monte Carlo, lo que simplifica considerablemente el manejo de fuentes en \texttt{OpenMC}.

% Este enfoque permite generar partículas para una siguiente simulación Monte Carlo. 

A través de este proceso es posible remuestrear nuevas partículas para la siguiente etapa de la simulación. Existen dos formas de integrar esta funcionalidad con \texttt{OpenMC}:

\begin{itemize}
    \item Una opción consiste en realizar, de forma \emph{offline}, el muestreo de una cantidad predeterminada de partículas y guardar sus propiedades en un archivo de partículas. Este archivo puede ser luego utilizado por \texttt{OpenMC} mediante su opción de simulación desde el modo de fuente \textit{FileSource} implementado en el programa.
    
    \item Alternativamente, se puede ejecutar \texttt{OpenMC} con una fuente del tipo \texttt{HistogramSource}, definida \textit{ad hoc} y configurada mediante un archivo \texttt{XML}. En este caso, \texttt{OpenMC} accede directamente al árbol de histogramas durante la simulación y genera cada partícula \emph{on-the-fly}, lo que reduce significativamente el uso de memoria y evita la necesidad de almacenar archivos de partículas intermedios voluminosos.
\end{itemize}

Esta funcionalidad fue incorporada en el contexto del presente trabajo, mediante la definición de una nueva clase de fuente en el código fuente de \texttt{OpenMC} en \texttt{C++}. Se desarrollaron las estructuras necesarias en \texttt{C} para efectuar el muestreo desde histogramas multidimensionales, y se integraron con la \texttt{API} de \texttt{OpenMC} en \texttt{Python}, permitiendo así que los histogramas puedan ser cargados y utilizados dentro del flujo habitual de simulación. El objetivo principal de esta implementación es facilitar el uso práctico de la herramienta.


\section{Implementación computacional}

La implementación computacional de la metodología descrita en este capítulo se encuentra desarrollada y documentada en el \textbf{Anexo A}. En particular, se presentan los códigos elaborados en \texttt{Python}, \texttt{C} y \texttt{C++} que posibilitan la generación y manejo de los histogramas multidimensionales generados a partir del archivo original de particulas. Dichos histogramas se configuran mediante parámetros específicos definidos por el usuario, tales como número y tipo de bines, así como el orden en que se procesan las variables del espacio de fases.

Asimismo, se documenta en detalle la integración realizada con \texttt{OpenMC}, incluyendo las modificaciones llevadas a cabo en su \textit{API} de \texttt{Python} y en su código fuente en \texttt{C++}, para permitir la generación \textit{on-the-fly} de partículas durante las simulaciones Monte Carlo subsecuentes.



% \section{Implementación computacional}
% La metodología descrita para estimar las distribuciones y correlaciones del \textit{trackfile} ha sido implementada en \texttt{Python}, debido a su leve/moderado costo computacional, creando una rutina que permite configurar parámetros tales como el número de bines y tipo de bineado para los histogramas macro y micro para cada variable, como así también el orden de procesamiento de las mismas. Esta rutina también ofrece la opción de insertar bordes manuales en los histogramas macro cuando se dispone de información previa que optimice la estimación.

% La estructura generada (árbol jerárquico de histogramas multidimensionales) es almacenada en un archivo \texttt{XML}, formato ideal debido a la estructura tipo árbol del dato generado. Posteriormente, esta información es utilizada directamente como fuente en simulaciones Monte Carlo subsecuentes en \texttt{OpenMC}, mediante modificaciones específicas realizadas tanto en su \textit{API} de \texttt{Python} como en su código fuente en \texttt{C++}.

% % \section{Conexión con la implementación computacional}
% La metodología expuesta fue implementada mediante códigos desarrollados en \texttt{Python}, \texttt{C} y \texttt{C++}, integrados específicamente dentro de los entornos de simulación Monte Carlo \texttt{OpenMC} y \texttt{KDSource}. La implementación detallada y comentada de estos códigos se presenta en el \textbf{Anexo A}, mostrando de forma explícita cómo se lleva a la práctica el proceso descrito anteriormente.

% En los capítulos siguientes el desempeño del método descrito será evaluado para cuantificar el grado en que el método conserva las propiedades originales del \textit{trackfile} registrado.

% \chapter{Evaluación del Método mediante Resampleo en \emph{Trackfiles}}
\label{chap:evaluacion-trackfiles}

\paragraph{Configuración 2: binning adaptativo (adaptive/adaptive).}  
Al aplicar binning adaptativo tanto en los macrogrupos como microgrupos, se optimiza la ubicación de los bordes para reflejar mejor la densidad local. En letargia se observa una mejora sustancial en la representación de los valores tipo delta, aunque persisten irregularidades en las zonas de letargia alta. Las variables espaciales muestran una mayor suavidad local sin perder detalle global.

Las métricas KL mejoran drásticamente respecto del caso uniforme:

\begin{center}
\begin{tabular}{lcc}
\toprule
\textbf{Configuración} & $\sum$KL 1D [nats] & $\sum$KL 2D [nats] \\
\midrule
Adaptive / Adaptive & 0.533 & 10.15 \\
\bottomrule
\end{tabular}
\end{center}

\paragraph{Configuración 3: macrobinning adaptativo, microbinning uniforme.}  
Esta variante mejora parcialmente la resolución estructural de la distribución, especialmente en 2D, pero mantiene el error en 1D cuando hay discontinuidades abruptas. Esto se refleja en un KL total intermedio:

\begin{center}
\begin{tabular}{lcc}
\toprule
\textbf{Configuración} & $\sum$KL 1D [nats] & $\sum$KL 2D [nats] \\
\midrule
Adaptive / Equal & 1.087 & 12.56 \\
\bottomrule
\end{tabular}
\end{center}

\paragraph{Configuración 4: macrobinning uniforme, microbinning adaptativo.}  
Esta última configuración aún no ha sido evaluada en detalle, pero se espera que muestre un comportamiento intermedio entre los casos previos. Se incluirá su análisis en una versión posterior.



\section{Resultados con Trackfile 1}
\label{sec:resultados-track1}

En esta sección se presentan los resultados obtenidos al aplicar el método sobre el primer archivo de tracks. El análisis se realizó utilizando múltiples configuraciones, variando el orden de las variables, la cantidad de macrogrupos y microgrupos.

Se muestran:

\begin{itemize}
    \item Distribuciones de energía, posición y dirección reconstruidas y su comparación con las originales.
    \item Curvas de error relativo por variable.
    \item Mapas de error relativo 2D para las correlaciones seleccionadas.
    \item Tabla con valores de divergencia KL para distintas configuraciones.
\end{itemize}

Estos resultados permiten discutir la sensibilidad del método a cada uno de los parámetros de entrada y establecer lineamientos para su elección óptima.

\subsection{Comparación entre esquema adaptativo y de ancho constante}
\label{subsec:adaptativo-vs-constante}

Como parte del análisis sobre el primer trackfile, se estudió el impacto del esquema de discretización utilizado en los macro y microgrupos. Se comparó explícitamente el rendimiento del esquema adaptativo frente a un esquema de histogramas con ancho constante.

El esquema adaptativo asigna mayor resolución a regiones del espacio de fases con alta estadística, y menor resolución donde los datos son escasos. Esto permite representar con mayor fidelidad las distribuciones sin amplificar el ruido estadístico.

Los resultados muestran una mejora significativa en la reconstrucción cuando se utiliza el esquema adaptativo. Esto se evidencia tanto en los gráficos de error relativo como en los valores de la divergencia KL, donde se observa una reducción consistente al aplicar el bineado adaptativo.

Esta comparación resalta la importancia de emplear técnicas de bineado adaptativo como herramienta para preservar la calidad del muestreo, especialmente en regiones con estructuras finas y alta variabilidad estadística.

\section{Resultados con Trackfile 2}
\label{sec:resultados-track2}

Se repitió el procedimiento metodológico sobre un segundo archivo de tracks con características geométricas y espectrales diferentes al primero, lo cual permite poner a prueba la generalidad del método propuesto.

Al igual que en el caso anterior, se analizaron distintas configuraciones de discretización jerárquica y orden de variables, prestando especial atención a:

\begin{itemize}
    \item La estabilidad del método ante distribuciones menos suaves o más concentradas espacialmente.
    \item La sensibilidad de las correlaciones bidimensionales al cambio en el esquema de macrogrupos.
    \item El comportamiento de la divergencia KL como función de la resolución utilizada.
\end{itemize}

Los gráficos comparativos muestran una buena reconstrucción general de las distribuciones 1D, aunque se observaron mayores errores relativos en regiones de baja estadística. En cuanto a las correlaciones, se destaca nuevamente la importancia de evitar configuraciones con fragmentación excesiva en las últimas variables del árbol.

Se presenta a continuación una selección representativa de los resultados gráficos y numéricos obtenidos, incluyendo:

\begin{itemize}
    \item Plots de distribuciones originales vs. reconstruidas.
    \item Mapas de error relativo 2D para variables espaciales y angulares.
    \item Tabla comparativa de divergencias KL.
\end{itemize}

\begin{table}[H]
    \centering
    \caption{Divergencia KL para distintas configuraciones en Trackfile 2.}
    \begin{tabular}{lccc}
        \toprule
        \textbf{Configuración} & \textbf{Macrogrupos} & \textbf{Microgrupos} & \textbf{KL Divergence} \\
        \midrule
        Orden A & [8,8,8,8] & [80,80,80,80] & 0.021 \\
        Orden B & [6,7,8,9] & [60,70,80,90] & 0.016 \\
        Orden C & [9,8,7,6] & [100,80,60,40] & 0.019 \\
        \bottomrule
    \end{tabular}
    \label{tab:trackfile2_kl}
\end{table}


\section{Resultados con Trackfile 3}
\label{sec:resultados-track3}

Finalmente, se aplicó el método sobre un tercer archivo de tracks con características mixtas, presentando una distribución energética más extendida y un patrón angular complejo, lo cual representa un desafío adicional para el muestreo jerárquico.

Se realizaron pruebas similares a las anteriores, enfocándose en:

\begin{itemize}
    \item Evaluar la robustez del método frente a distribuciones con múltiples picos o simetrías rotas.
    \item Observar el efecto del esquema de macrogrupos decreciente en variables angulares.
    \item Validar si se mantiene la tendencia en la divergencia KL al aplicar bineado adaptativo.
\end{itemize}

Los resultados obtenidos son consistentes con los de los otros casos, aunque se observaron diferencias notables en la reconstrucción de las variables angulares cuando éstas se encontraban en posiciones tempranas dentro del árbol, lo cual sugiere una posible pérdida de fidelidad debido a fragmentación excesiva.

\begin{table}[H]
    \centering
    \caption{Divergencia KL en Trackfile 3 para diferentes combinaciones.}
    \begin{tabular}{lccc}
        \toprule
        \textbf{Configuración} & \textbf{Macrogrupos} & \textbf{Microgrupos} & \textbf{KL Divergence} \\
        \midrule
        Esquema uniforme & [6,6,6,6] & [60,60,60,60] & 0.024 \\
        Esquema creciente & [6,7,8,9] & [60,70,80,90] & 0.017 \\
        Esquema adaptativo & Variable & Adaptativo & 0.012 \\
        \bottomrule
    \end{tabular}
    \label{tab:trackfile3_kl}
\end{table}

Estos resultados reafirman la utilidad del esquema adaptativo en escenarios complejos, donde las estructuras locales en el espacio de fases requieren una resolución flexible para ser correctamente representadas.



\chapter{Aplicación al Caso del Canal de Vacío en Agua}

\section{Descripción de la geometría y condiciones de frontera}

El caso de estudio planteado consiste en un canal de vacío embebido en un bloque de agua liviana, configurado de forma tal que reproduce condiciones similares a una guía de neutrones, pero en un entorno simplificado y controlado. El sistema se compone de un paralelepípedo de agua de dimensiones $15\,\text{cm} \times 15\,\text{cm} \times 100\,\text{cm}$, dentro del cual se ubica un canal interno de vacío de sección $3\,\text{cm} \times 3\,\text{cm}$ orientado a lo largo del eje $z$.

% \begin{figure}[H]
%     \centering
%     \includegraphics[width=0.6\textwidth]{figuras/canal_vacio_esquema.pdf}
%     \caption{Esquema de la geometría del sistema: bloque de agua con canal de vacío central.}
%     \label{fig:canal-vacio}
% \end{figure}

Se define una fuente plana ubicada en la cara de entrada del sistema ($z = 0$), compuesta por neutrones monoenergéticos de $E = 1\,\text{MeV}$, perfectamente colimados a lo largo del eje $z$ (i.e., con $\mu = 1$). Esta configuración da lugar a dos poblaciones de neutrones marcadamente diferentes: aquellos que atraviesan el canal de vacío sin colisionar, manteniendo su energía y dirección originales, y aquellos que interactúan con el moderador, sufriendo pérdida de energía y dispersión angular.

\section{Problemas al utilizar histogramas con macrogrupos uniformes}

Se implementó una simulación con desacople en una superficie transversal ubicada en $z = z_0$, registrando las partículas que la atraviesan. El muestreo posterior basado en histogramas con macrogrupos de ancho constante mostró deficiencias notables: algunos macrogrupos intersectaban simultáneamente regiones de agua y vacío, mezclando partículas con características físicas disímiles. Esta superposición genera una pérdida importante en la correlación entre variables, particularmente entre la dirección ($\mu$) y la posición transversal ($x$, $y$).

% \begin{figure}[H]
%     \centering
%     \includegraphics[width=0.45\textwidth]{figuras/mu_x_uniforme.pdf}
%     \includegraphics[width=0.45\textwidth]{figuras/mu_x_correlacion_perdida.pdf}
%     \caption{Ejemplo de pérdida de correlación $\mu$ vs.\ $x$ al utilizar macrogrupos uniformes.}
%     \label{fig:mu-x-perdida}
% \end{figure}

\section{Mejora mediante bordes de macrogrupos definidos manualmente}

Para mitigar esta pérdida de información, se incorporó la posibilidad de definir manualmente los bordes de los macrogrupos en las variables críticas. Esto permitió aislar espacialmente la región correspondiente al canal de vacío, evitando que neutrones colimados se mezclen con partículas moderadas.

Se analizaron tres configuraciones diferentes:

\begin{itemize}
    \item \textbf{Caso A}: definición manual en las variables $x$ e $y$ (espaciales).
    \item \textbf{Caso B}: definición manual en letargia y $\mu$ (energética y direccional).
    \item \textbf{Caso C}: bordes definidos en las cuatro variables ($x$, $y$, $\mu$, letargia).
\end{itemize}

% \begin{figure}[H]
%     \centering
%     \includegraphics[width=0.7\textwidth]{figuras/macrobordes_configuraciones.pdf}
%     \caption{Esquemas de segmentación para los casos A, B y C.}
%     \label{fig:macrobordes}
% \end{figure}

\begin{table}[H]
    \centering
    \caption{Divergencia KL entre distribución original y resampleada en cada caso.}
    \label{tab:KL-bordes}
    \begin{tabular}{lccc}
        \toprule
        Variable & Caso A & Caso B & Caso C \\
        \midrule
        Letargia & 0.112 & 0.045 & \textbf{0.028} \\
        $\mu$     & 0.295 & 0.071 & \textbf{0.031} \\
        $x$       & 0.148 & 0.121 & \textbf{0.037} \\
        $y$       & 0.159 & 0.130 & \textbf{0.035} \\
        \bottomrule
    \end{tabular}
\end{table}

\section{Resultados con histogramas adaptativos}

Posteriormente se aplicó un esquema de histogramas adaptativos, donde la subdivisión de los macrogrupos fue determinada de forma automática en función de la densidad estadística. Esta técnica permitió una segmentación más eficiente, sin requerir intervención manual del usuario.

Los resultados mostraron una mejora significativa en la reconstrucción de las distribuciones 1D y 2D, así como en las métricas de error globales como la divergencia de Kullback-Leibler (KL) y el error medio absoluto.

% \begin{figure}[H]
%     \centering
%     \includegraphics[width=0.48\textwidth]{figuras/rel_error_adaptativo_2D.pdf}
%     \includegraphics[width=0.48\textwidth]{figuras/KL_vs_metodo.pdf}
%     \caption{Izq: error relativo en plano $\mu$ vs.\ letargia. Der: comparación de KL entre métodos.}
%     \label{fig:adaptativo-resultados}
% \end{figure}

\section{Validación de tallies y aplicación de técnicas de reducción de varianza}

Para validar los resultados se evaluaron distintas magnitudes físicas a lo largo del eje del sistema:

\begin{itemize}
    \item Flujo escalar en secciones transversales: total, en agua, y en vacío.
    \item Espectro energético sobre una superficie de tally a $z = 80\,\text{cm}$.
    \item Corriente en dirección $z$ sobre planos intermedios.
\end{itemize}

Se aplicó reducción de varianza mediante weight windows generados con OpenMC, especialmente en regiones con moderación intensa, lo cual mejoró la estadística de tallies en el agua sin alterar el resultado global.

% \begin{figure}[H]
%     \centering
%     \includegraphics[width=0.65\textwidth]{figuras/flujo_vs_z_comparacion.pdf}
%     \caption{Flujo escalar promedio a lo largo del canal. Comparación entre simulación original y reconstruida.}
%     \label{fig:flujo-vs-z}
% \end{figure}

\section{Síntesis y conclusiones}

El caso del canal de vacío embebido en agua permitió poner en evidencia los desafíos que presentan las técnicas de muestreo cuando coexisten poblaciones de partículas con comportamientos disímiles. Se comprobó que:

\begin{itemize}
    \item La segmentación espacial y direccional es crucial para preservar correlaciones en el muestreo.
    \item Los histogramas adaptativos constituyen una alternativa robusta y automática frente a configuraciones manuales.
    \item El método propuesto reproduce con alta fidelidad los resultados de flujo, espectro y corriente.
\end{itemize}

Este caso sirve como referencia para futuras aplicaciones en geometrías más complejas donde también existan discontinuidades materiales o comportamientos multi-modales del campo de neutrones.


\chapter{Validación experimental: Conducto Nº5 del reactor RA-6}
% \label{chap:validacion-ra6}

En el presente capítulo se validará el método de remuestreo de partículas mediante histogramas multidimensionales desarrollado en este trabajo. Para ello, se aplicará dicho método al conducto Nº5 del reactor RA-6, el reactor nuclear de investigación tipo pileta ubicado en el Centro Atómico Bariloche (CAB), Argentina.

El RA-6 cuenta con cinco conductos de extracción de neutrones diseñados para transportar haces desde el núcleo hacia diferentes instalaciones experimentales. En particular, el Departamento de Neutrones del CAB ha desarrollado espectrometría basada en la técnica de tiempo de vuelo (TdV) \cite{Schmidt2021Chopper} sobre el conducto Nº5. Esto se realiza utilizando un componente denominado \textit{chopper}, cuya función es pulsar el haz de neutrones. Midiendo el tiempo que tardan los neutrones en viajar desde el \textit{chopper} hasta un banco de detectores de \textsuperscript{3}He, se obtiene el espectro de energía del haz mediante TdV.

A su vez, el Departamento de Neutrones ha realizado simulaciones Monte Carlo del RA-6 utilizando el código \texttt{OpenMC} con el objetivo de estimar la distribución espectral en el banco de detectores. Sin embargo, debido a la baja probabilidad de que un neutrón simulado desde el núcleo alcance dicha región, una simulación directa resulta computacionalmente inviable con los recursos disponibles.

Para superar esta limitación, se empleó una segmentación geométrica del problema. En una primera simulación desde núcleo, el Departamento de Neutrones registró un archivo de partículas en la entrada del conducto Nº5. Con este archivo de partículas se generó una fuente distribucional con el método de histogramas multidimensionales. Esta fuente se utilizó como entrada en una segunda simulación de \texttt{OpenMC}, que modela exclusivamente el conducto Nº5 hasta el banco de detectores.

Este enfoque permite generar una mayor cantidad de partículas en la entrada del conducto Nº5 de las que se registró originalmente en el archivo de partículas. Esto permite incrementar la estadística en los detectores sin necesidad de simular desde el núcleo. 

De este modo, se posibilita una comparación entre los resultados obtenidos aplicando el método de remuestreo utilizando histogramas multidimensionales y las mediciones experimentales realizadas por el Departamento de Neutrones, permitiendo validar el método implementado.

En las secciones siguientes se describen la geometría del reactor y del conducto Nº5, y los resultados obtenidos de flujo neutrónico y espectro de energía.

\section{Descripción del reactor RA-6 y instalación experimental del conducto Nº5}

El núcleo del reactor RA-6 está compuesto por veinte elementos combustibles del tipo placa de aluminio \textit{meat} de siliciuro de uranio enriquecido al 19,70\%. Este conjunto, cuya altura activa alcanza los 61,9~cm, se encuentra sumergido en una pileta de agua liviana que actúa simultáneamente como moderador, refrigerante, reflector axial y blindaje biológico. Lateralmente, el reflector está constituido por bloques de grafito. El núcleo opera a una potencia de 1~MW térmico y está alojado en una pileta cilíndrica de 2,4~m de diámetro, rodeada a su vez por un blindaje biológico de hormigón pesado con forma octogonal. En la Figura \ref{fig:esquema-nucleo} se presenta un esquema representativo del núcleo y sus alrededores.

\begin{figure}[h]
\centering
\includegraphics[width=0.75\textwidth]{ra6.png}
\caption[Esquema representativo del núcleo del RA-6. En el mismo se observa la disposición de los elementos combustibles, el conducto Nº1 (inferior) y el conducto Nº5 (superior).]{Esquema representativo del núcleo del RA-6. En el mismo se observa la disposición de los elementos combustibles, el conducto Nº1 (inferior) y el conducto Nº5 (superior) \cite{Schmidt2021Chopper}.}
\label{fig:esquema-nucleo}
\end{figure}

El reactor cuenta con cinco conductos de extracción de neutrones destinados a experimentos neutrónicos. En la Figura~\ref{fig:esquema-nucleo} se destacan dos de ellos: el conducto Nº1 y el conducto Nº5. En particular, el conducto Nº5 se orienta hacia una zona del núcleo en la que se encuentran ubicadas un bloque de grafito y un bloque de agua. Dado que no apunta directamente hacia los elementos combustibles, el espectro resultante en este canal está compuesto mayoritariamente por neutrones térmicos.

El conducto Nº5 consiste en un cilindro de acero de 5~cm de radio a la entrada que aumenta a 7,5~cm de radio en el tramo final. Este atraviesa la pileta del reactor y el blindaje biológico. En su interior, en la región de sección de 7,5~cm de radio, se encuentra un colimador constituido por secciones alternadas de plomo y parafina borada. Además, las dos primeras secciones del colimador contienen un filtro de bismuto, utilizado para atenuar la componente de radiación $\gamma$ del haz. Esta atenuación resulta fundamental en experimentos, ya que permite reducir el fondo de radiación $\gamma$ que puede saturar los detectores utilizados. La Figura \ref{fig:conducto} muestra un esquema detallado del conducto Nº5 y de su sistema de colimación, como así también el banco de detectores utilizados en la medición experimental.

\begin{figure}[h]
\centering
\includegraphics[width=0.9\textwidth]{conducto.png}
\caption[Esquema representativo del conducto Nº5 del RA-6 y de la instalación experimental asociada al \textit{chopper}.]{Esquema representativo del conducto Nº5 del RA-6 y de la instalación experimental asociada al \textit{chopper} \cite{DeptoNeutronesCAB2025}.}
\label{fig:conducto}
\end{figure}

Este conducto está asociado a un componente experimental denominado \textit{chopper}, diseñado para la generación de haces pulsados de neutrones. El \textit{chopper} consiste en un disco rotatorio de material absorbente con una ranura, el cual se posiciona a la salida del conducto. Al girar a velocidad constante, la ranura permite periódicamente el paso de neutrones cuando se encuentra alineada con el eje del haz. De esta manera, se obtienen pulsos neutrónicos que pueden ser utilizados para realizar espectrometría mediante la técnica de tiempo de vuelo (TdV). La Figura \ref{fig:chopper} ilustra esquemáticamente el mecanismo de funcionamiento del disco \textit{chopper}.

\begin{figure}[h]
\centering
\includegraphics[width=0.5\textwidth]{DISCO_CHOPPER.png}
\caption[Esquema representativo del disco rotatorio del \textit{chopper} empleado en el conducto Nº5. La ranura permite el paso periódico de neutrones, generando un haz pulsado.]{Esquema representativo del disco rotatorio del \textit{chopper} empleado en el conducto Nº5. La ranura permite el paso periódico de neutrones, generando un haz pulsado \cite{Schmidt2021Chopper}.}
\label{fig:chopper}
\end{figure}

Para la implementación experimental de esta técnica, se dispone un banco de cinco detectores de \textsuperscript{3}He a una distancia conocida de la salida del conducto. Estos detectores registran el tiempo de llegada de cada neutrón, a partir del cual se calcula su energía cinética mediante la expresión:

\begin{equation}
E = \frac{1}{2} m_n v^2,
\end{equation}

donde $E$ es la energía del neutrón, $m_n$ su masa, y $v$ la velocidad obtenida como el cociente entre la distancia conocida y el tiempo medido.

Para reconstruir correctamente el espectro neutrónico, es necesario corregir la señal detectada. En primer lugar, se debe restar un fondo constante asociado a la componente gamma y a neutrones que atraviesan el \textit{chopper} fuera de fase. Adicionalmente, se aplican correcciones por el tiempo muerto del sistema de adquisición de datos y por la eficiencia de detección de los detectores de \textsuperscript{3}He.

Finalmente, detrás del banco de detectores se instala un blindaje adicional que permite blindar el haz y proteger las áreas experimentales circundantes.

\section{Procesamiento del archivo de partículas}

El Departamento de Neutrones del Centro Atómico Bariloche proporcionó un archivo de partículas que contiene información únicamente sobre los neutrones que ingresan al conducto Nº5, filtrando los fotones. Este archivo fue generado a partir de una simulación del núcleo completo del reactor RA-6 utilizando el código \texttt{OpenMC}. La simulación consistió en un total de $10^{10}$ neutrones, de los cuales 41245 fueron registrados en la entrada del conducto Nº5.

Cabe destacar que esta simulación fue realizada con un esquema de reducción de varianza, lo que implica que muchas partículas del archivo presentan un peso estadístico menor a la unidad. La suma total de los pesos estadísticos de las partículas registradas fue 22182.

En la Figura~\ref{fig:conducto-XY} se muestra la distribución espacial de las partículas registradas en el plano $XY$. Puede observarse que las posiciones conforman un círculo de radio 5~cm, correspondiente a la sección transversal del conducto Nº5. La distribución es aproximadamente uniforme dentro del círculo.

\begin{figure}[h]
\centering
\includegraphics[width=0.75\textwidth]{conducto_XY.png}
\caption{Distribución espacial de las partículas del archivo original en el plano $XY$.}
\label{fig:conducto-XY}
\end{figure}

La Figura~\ref{fig:conducto-let} presenta la distribución de letargía de las partículas registradas. Se observan dos regiones de acumulación: una alrededor de $\ln(E_0/E) \approx 2$, correspondiente a energías típicas de fisión, y otra de mayor intensidad en la región térmica, centrada en $\ln(E_0/E) \approx 19$, donde $E_0 = 20$~MeV.

\begin{figure}[h]
\centering
\includegraphics[width=\textwidth]{conducto_let.png}
\caption{Distribución de letargía de las partículas del archivo original.}
\label{fig:conducto-let}
\end{figure}

A partir de este archivo de partículas, se generó una fuente distribucional mediante el método de histogramas multidimensionales desarrollado en este trabajo. La configuración adoptada para la construcción de la fuente distribucional fue la siguiente:

\begin{itemize}
\item \textbf{Orden de procesamiento}: \texttt{[letargía, X, Y, $\mu$, $\phi$]}
\item \textbf{Método de \textit{bineado} micro y macro}: \textit{bineado} adaptativo.
\item \textbf{Número de histogramas macro}: \texttt{[3, 7, 4, 4]}
\item \textbf{Número de histogramas micro}: \texttt{[75, 18, 18, 15, 10]}
\end{itemize}

Se asignó una mayor resolución a la variable letargía, dado que constituye el objetivo principal de análisis en este capítulo, centrado en la comparación espectral mediante la técnica de tiempo de vuelo. Además, se destinó una mayor cantidad de divisiones macro a la coordenada $X$ para mejorar la representación del contorno circular del conducto. Se verificó que al reducir la resolución en dicha coordenada, aparecían patrones discretos artificiales en el plano $XY$, producto de una segmentación insuficiente del dominio espacial.

La Figura~\ref{fig:conducto-comparacion-XY} muestra la distribución en el plano $XY$ de las partículas generadas por la fuente distribucional. Puede observarse que la forma circular de radio 5~cm se reproduce adecuadamente, aunque la discretización asociada a los histogramas macro introduce estructuras visibles en los bordes.

\begin{figure}[h]
\centering
\includegraphics[width=0.75\textwidth]{conducto_comparacion_XY.png}
\caption{Distribución espacial de las partículas de la fuente distribucional en el plano $XY$.}
\label{fig:conducto-comparacion-XY}
\end{figure}

En la Figura~\ref{fig:conducto-comparacion-let} se presenta la comparación entre la distribución de letargía del archivo original y la correspondiente a la fuente generada. La mayor resolución elegida para el eje de letargía permite representar en detalle la zona térmica, a costa de reproducir parte del ruido estadístico presente en la región epitérmica.

\begin{figure}[h]
\centering
\includegraphics[width=\textwidth]{conducto_comparacion_let.png}
\caption{Comparación de la distribución de letargía entre el archivo original y la fuente distribucional.}
\label{fig:conducto-comparacion-let}
\end{figure}

\section{Resultados de simulación en \texttt{OpenMC} y comparación experimental}

A partir de la fuente distribucional generada mediante histogramas multidimensionales, se llevó a cabo una segunda simulación del conducto Nº5, esta vez utilizando dicha fuente como entrada en el código \texttt{OpenMC}. La simulación constó de un total de $5 \times 10^9$ partículas. Con el objetivo de aumentar la probabilidad de detección y mejorar la eficiencia estadística, se implementó un esquema de reducción de varianza de absorción implícita.

Durante la simulación se registró el flujo neutrónico en todo el modelo, así como el espectro de neutrones detectados en el banco de detectores. El espectro simulado fue luego comparado con los datos experimentales obtenidos mediante la técnica de tiempo de vuelo (TdV) en el RA-6 por el Departamento de Neutrones.

\subsection{Distribución espacial del flujo}

En la Figura~\ref{fig:conducto-flujo} se presenta la distribución espacial del flujo neutrónico obtenido en el modelo del conducto Nº5. Puede observarse que se logra una acumulación estadística significativa en la región del banco de detectores. Asimismo, se evidencia una fuerte atenuación del flujo a lo largo del conducto y el colimador. Esta atenuación refleja la baja probabilidad de que los neutrones atraviesen el conducto y lleguen al banco de detectores.

Como se ha mostrado en trabajos relacionados sobre el conducto Nº5 del RA-6 \cite{Schmidt2021Chopper}, simulaciones realizadas directamente desde el núcleo con cantidades de partículas del orden de $10^{10}$ no logran atravesar el colimador con suficiente estadística, resultando en flujos despreciables en la región de detectores. En contraste, en la Figura~\ref{fig:conducto-flujo} puede apreciarse que, al aplicar el desacople geométrico, se ha obtenido un flujo con buena estadística en toda la zona de interés, utilizando una cantidad de partículas del mismo orden.

\begin{figure}[h]
\centering
\includegraphics[width=0.7\textwidth]{conducto_flujo.png}
\caption{Distribución del flujo neutrónico en el modelo del conducto Nº5.}
\label{fig:conducto-flujo}
\end{figure}

Este comportamiento justifica la estrategia de desacoplar geométricamente la simulación del conducto respecto del núcleo del reactor. Simular directamente desde el núcleo implicaría combinar la baja probabilidad de que los neutrones alcancen la entrada del conducto con la también baja probabilidad de que atraviesen el colimador y lleguen al banco de detectores, resultando en una estadística reducida. En cambio, al utilizar la fuente distribucional en la entrada del conducto como fuente, se logra concentrar los recursos computacionales en la región de interés, obteniendo una estadística adecuada para el análisis espectral y mejorando los resultados.

\subsection{Comparación con la medición experimental}

La Figura~\ref{fig:conducto-espectro} muestra el espectro neutrónico obtenido en el banco de detectores a partir de la simulación con fuente distribucional, en comparación con el espectro medido experimentalmente mediante TdV. Se observa concordancia a partir de energías del orden de 0.01~eV. No obstante, por debajo de ese umbral, el flujo simulado subestima al medido experimentalmente.

\begin{figure}[h]
\centering
\includegraphics[width=\textwidth]{conducto_espectro.png}
\caption{Comparación del espectro de neutrones obtenido en el banco de detectores en la simulación de \texttt{OpenMC} con la medición experimental realizada por TdV por el Departamento de Neutrones del CAB. Para ambas curvas se grafican los errores estadísticos.}
\label{fig:conducto-espectro}
\end{figure}

Esta discrepancia en la región de bajas energías puede atribuirse principalmente a cuatro factores:

\begin{itemize}
\item \textbf{Limitaciones en la representación espectral de la fuente}. Tal como se observa en la Figura~\ref{fig:conducto-comparacion-let}, existe diferencia entre el archivo original y la fuente distribucional en la región de mayor letargía, correspondiente a las energías menores a 0.01~eV. Esta desviación se origina en la discretización inherente al método de histogramas.
\item \textbf{Tratamiento del filtro de bismuto en la simulación}. En la modelización realizada con \texttt{OpenMC}, se emplean secciones eficaces correspondientes a átomos de bismuto como gas libre. Sin embargo, el filtro real está construido con bismuto cristalino, cuyas secciones eficaces difieren a bajas energías \cite{Mishra2006PULSTAR}. Mientras que para energías superiores a 0.1~eV ambas secciones eficaces son equivalentes, por debajo de ese valor la sección eficaz del bismuto cristalino es menor. La máxima discrepancia se presenta en torno a 1~meV, donde la sección eficaz del bismuto cristalino representa apenas el 10\% de la correspondiente al bismuto libre. Este efecto conduce a una atenuación del flujo térmico simulado, explicando la subestimación observada en la comparación con los datos experimentales.
\item \textbf{Tratamiento experimental de los datos medidos}. La medición experimental utilizada como referencia ha sido obtenida a partir de mediciones realizadas con detectores de ${}^3$He, y requiere un proceso de postratamiento. Este proceso incluye correcciones por tiempo muerto de los detectores, eficiencia de detección de los detectores de ${}^3$He y por nivel de fondo. En particular, la estimación y substracción del nivel de fondo puede introducir variabilidad en la forma final del espectro, especialmente en la región correspondiente a menores energías. Como es esta la región del espectro donde se registra la mayor discrepancia respecto a la simulación, no puede descartarse que parte del desvío observado esté asociado a incertidumbres propias del procedimiento de corrección experimental aplicado.
\item \textbf{Incerteza de la simulación computacional}. Toda simulación Monte Carlo está sujeta a una serie de fuentes de error que deben ser consideradas al interpretar los resultados. Entre ellas se incluyen las simplificaciones geométricas necesarias al modelar la instalación experimental, las incertidumbres en las secciones eficaces utilizadas por el código \texttt{OpenMC}, y el error estadístico asociado al número finito de partículas simuladas. Adicionalmente, el uso de un archivo de partículas finito para la construcción de la fuente distribucional introduce su propio error estadístico, que se propaga al remuestreo posterior. Finalmente, el método de generación de fuentes mediante histogramas multidimensionales introduce un error adicional debido a la discretización del espacio de fases, que puede afectar la precisión en la reconstrucción de distribuciones continuas.

\end{itemize}

A pesar de estas limitaciones, los resultados obtenidos permiten validar el enfoque implementado, demostrando que la fuente distribucional construida es capaz de reproducir adecuadamente el comportamiento espectral de los neutrones en el banco de detectores en el rango de interés térmico.


\chapter{Conclusiones}
\label{chap:conclusiones}

En este trabajo se desarrolló e implementó un método para generar fuentes distribucionales para simulaciones Monte Carlo basado en el uso de histogramas multidimensionales. La implementación se realizó dentro del código \texttt{KDSource}.

La metodología desarrollada fue validada mediante pruebas analíticas y contra mediciones experimentales, confirmando su capacidad para reproducir distribuciones en el espacio de fases. En particular, se mostró que la estructura de histogramas multidimensionales empleada logra preservar satisfactoriamente las correlaciones de las variables del espacio de fases de las partículas.

Adicionalmente, se implementó un esquema de bineado adaptativo, el cual permitió captar con precisión regiones donde las distribuciones exhiben cambios bruscos, deltas o bordes definidos, incrementando la resolución local del histograma de manera autónoma. Este esquema resultó especialmente útil en situaciones donde las distribuciones presentan estructuras complejas difíciles de caracterizar mediante discretizaciones uniformes.

Por otro lado, se realizó una implementación específica de remuestreo \textit{on-the-fly} dentro del código fuente de \texttt{OpenMC}. Esta modificación permitió el remuestreo de partículas durante la ejecución de la simulación, sin la necesidad de generar listas intermedias de partículas. La técnica se aplicó en el caso del conducto Nº5 del reactor RA-6, obteniendo resultados en concordancia con datos experimentales disponibles, validando así el método desarrollado.

\subsection*{Trabajo futuro}

A partir de los resultados obtenidos y las capacidades mostradas por el método implementado, se identifican diversas líneas futuras de investigación que podrían extender su alcance y utilidad práctica. Una primera línea consiste en la generalización del método hacia otras geometrías más complejas y variadas, más allá de las superficies planas consideradas en este trabajo. Esto incluiría, por ejemplo, fuentes distribuidas sobre superficies curvas o laterales de un conducto.

Adicionalmente, sería valioso incorporar técnicas que permitan un refinamiento local dirigido por el usuario. En particular, podría implementarse la capacidad de especificar manualmente un mayor número de bines en rangos específicos para ciertas variables, mejorando selectivamente la resolución de la distribución sin afectar la estructura general de macrogrupos utilizada. Esta capacidad permitiría a los usuarios optimizar el detalle local en la generación de fuentes, adaptándose de manera flexible a distintos escenarios y necesidades de simulación.

Finalmente, dado que el método desarrollado ha sido implementado únicamente para neutrones, sería relevante extender su aplicación a otras partículas, como fotones, ampliando así sus posibles aplicaciones en simulaciones Monte Carlo.


\appendix
% \appendix
\chapter{Implementación computacional del método}
\label{app:A}

% \section{Código detallado en Python/C++}
% \label{app:codigo}
% % Incluir listings o referencias a archivos externos

% \section{Pseudocódigo y comentarios}
% \label{app:pseudocodigo}

% \section{Archivos de configuración XML}
% \label{app:xml}

% \section{Flujo de trabajo práctico}
% \label{app:flujo}
%========================================================


% \appendix
\chapter{Ejemplificación del bineado adaptativo}
\label{app:B}

En este apéndice se presenta una serie de ejemplos ilustrativos que permiten visualizar el funcionamiento del algoritmo de bineado adaptativo descrito en el Capítulo~\ref{cap:metodo_histogramas}. La variable seleccionada para la demostración es la \textit{letargía} de un archivo de partículas, que se presenta en la Figura~\ref{fig:B_letargia}. La distribución de letargía en este caso exhibe tres regiones distintivas:

\begin{figure}[H]
    \centering
    \includegraphics[width=\textwidth]{B_letargia.png}
    \caption{Distribución de letargía utilizada en la ejemplificación.}
    \label{fig:B_letargia}
\end{figure}

\begin{itemize}
    \item Una delta en $1~MeV$ asociada a una fuente monoenergética.
    \item Una región aproximadamente constante en la región epitérmico.
    \item Una acumulación en la región térmica.
\end{itemize}

Se parte de la distribución original de referencia, obtenida con alta resolución de bines, y se aplica el algoritmo de bineado adaptativo siguiendo el criterio de máxima discrepancia en la función de distribución acumulada (FDC). En cada iteración, se identifica el punto donde la diferencia absoluta entre la FDC original y la obtenida con el bineado actual es máxima, y se introduce allí un nuevo borde de bin. En este apendice se utiliza 1 bin como punto de partida, y se van añadiendo bines hasta alcanzar un total de 100 bines, lo que permite observar cómo el algoritmo refina la distribución a medida que se incrementa el número de bines.

En las siguientes secciones se presentan los resultados obtenidos para distintos números de bines: 1, 2, 3, 5, 10, 25 y 100. En cada caso, se muestran:

\begin{itemize}
    \item La distribución estimada con el bineado actual, comparada con la distribución de referencia.
    \item Las funciones de distribución acumulada (FDC) de ambas distribuciones.
    \item La diferencia absoluta entre ambas FDC, utilizada como criterio de refinamiento.
\end{itemize}

\section*{Caso con 1 bin}
\addcontentsline{toc}{section}{Caso con 1 bin}

En la Figura~\ref{fig:B_letargia_1} se muestra la distribución de letargía aproximada con un único bin. Dado que el histograma se construye con un único bin, representa el valor promedio de letargía sobre todo el dominio. En la Figura~\ref{fig:B_FDC_1} se presentan las FDCs de la distribución original de referencia y la obtenida con 1 bin, mientras que en la Figura~\ref{fig:B_FDCdiff_1} se muestra la diferencia absoluta entre ambas FDCs, que sirve como criterio para determinar la ubicación del nuevo bin en la siguiente iteración.

\begin{figure}[H]
    \centering
    \includegraphics[width=\textwidth]{B_letargia_1.png}
    \caption{Distribución de letargía aproximada con un único bin. Se observa que el histograma es constante, representando el valor promedio sobre todo el dominio.}
    \label{fig:B_letargia_1}
\end{figure}

\begin{figure}[H]
    \centering
    \begin{subfigure}[b]{0.46\textwidth}
        \includegraphics[width=\textwidth]{B_FDC_1.png}
        \caption{FDC original vs. FDC con 1 bin.}
        \label{fig:B_FDC_1}
    \end{subfigure}
    \hfill
    \begin{subfigure}[b]{0.46\textwidth}
        \includegraphics[width=\textwidth]{B_FDCdiff_1.png}
        \caption{Diferencia absoluta entre FDCs.}
        \label{fig:B_FDCdiff_1}
    \end{subfigure}
    \caption{Evaluación del criterio de refinamiento para el caso de 1 bin.}
    \label{fig:B_FDC_1_1}
\end{figure}

\section*{Caso con 2 bines}
\addcontentsline{toc}{section}{Caso con 2 bines}

En la Figura~\ref{fig:B_letargia_2} se muestra la distribución de letargía aproximada con 2 bines. En este caso, el histograma comienza a capturar la delta en $1~MeV$, pero con un bin de ancho considerable. En la Figura~\ref{fig:B_FDC_2} se presentan las FDCs de la distribución original de referencia y la obtenida con 2 bines, mientras que en la Figura~\ref{fig:B_FDCdiff_2} se muestra la diferencia absoluta entre ambas FDCs, que sirve como criterio para determinar la ubicación del nuevo bin en la siguiente iteración. Se observa que la posición del máximo en la Figura~\ref{fig:B_FDCdiff_2} ahora se encuentra con valor 0, producto de que se ubicó un bin en esa posición.

\begin{figure}[H]
    \centering
    \includegraphics[width=\textwidth]{B_letargia_2.png}
    \caption{Distribución de letargía aproximada con 2 bines. Se observa que el método comienza a capturar la delta en $1~MeV$, pero con un bin de ancho considerable.}
    \label{fig:B_letargia_2}
\end{figure}

\begin{figure}[H]
    \centering
    \begin{subfigure}[b]{0.46\textwidth}
        \includegraphics[width=\textwidth]{B_FDC_2.png}
        \caption{FDC original vs. FDC con 2 bines.}
        \label{fig:B_FDC_2}
    \end{subfigure}
    \hfill
    \begin{subfigure}[b]{0.46\textwidth}
        \includegraphics[width=\textwidth]{B_FDCdiff_2.png}
        \caption{Diferencia absoluta entre FDCs.}
        \label{fig:B_FDCdiff_2}
    \end{subfigure}
    \caption{Evaluación del criterio de refinamiento para el caso de 2 bines.}
    \label{fig:B_FDC_2_2}
\end{figure}

\section*{Caso con 3 bines}
\addcontentsline{toc}{section}{Caso con 3 bines}

En la Figura~\ref{fig:B_letargia_3} se muestra la distribución de letargía aproximada con 3 bines. En este caso, el histograma logra capturar la delta en $1~MeV$ con mayor precisión. En la Figura~\ref{fig:B_FDC_3} se presentan las FDCs de la distribución original de referencia y la obtenida con 3 bines, mientras que en la Figura~\ref{fig:B_FDCdiff_3} se muestra la diferencia absoluta entre ambas FDCs, que sirve como criterio para determinar la ubicación del nuevo bin en la siguiente iteración. \textbf{Nota para el corrector: aca en mas estoy repitiendo este parrafo para introducir las figuras y que no queden sin referencia. Podemos ver de cambiarlo si lo consideran necesario.}

\begin{figure}[H]
    \centering
    \includegraphics[width=\textwidth]{B_letargia_3.png}
    \caption{Distribución de letargía aproximada con 3 bines. Se observa que el método logra capturar la delta en $1~MeV$.}
    \label{fig:B_letargia_3}
\end{figure}

\begin{figure}[H]
    \centering
    \begin{subfigure}[b]{0.46\textwidth}
        \includegraphics[width=\textwidth]{B_FDC_3.png}
        \caption{FDC original vs. FDC con 3 bines.}
        \label{fig:B_FDC_3}
    \end{subfigure}
    \hfill
    \begin{subfigure}[b]{0.46\textwidth}
        \includegraphics[width=\textwidth]{B_FDCdiff_3.png}
        \caption{Diferencia absoluta entre FDCs.}
        \label{fig:B_FDCdiff_3}
    \end{subfigure}
    \caption{Evaluación del criterio de refinamiento para el caso de 3 bines.}
    \label{fig:B_FDC_3_3}
\end{figure}

\section*{Caso con 5 bines}
\addcontentsline{toc}{section}{Caso con 5 bines}

En la Figura~\ref{fig:B_letargia_5} se muestra la distribución de letargía aproximada con 5 bines. En este caso, el histograma comienza a capturar la distribución en la región de termalización. En la Figura~\ref{fig:B_FDC_5} se presentan las FDCs de la distribución original de referencia y la obtenida con 5 bines, mientras que en la Figura~\ref{fig:B_FDCdiff_5} se muestra la diferencia absoluta entre ambas FDCs, que sirve como criterio para determinar la ubicación del nuevo bin en la siguiente iteración.

\begin{figure}[H]
    \centering
    \includegraphics[width=\textwidth]{B_letargia_5.png}
    \caption{Distribución de letargía aproximada con 5 bines. Se observa que el método comienza a capturar la distribución en la región de termalización.}
    \label{fig:B_letargia_5}
\end{figure}

\begin{figure}[H]
    \centering
    \begin{subfigure}[b]{0.46\textwidth}
        \includegraphics[width=\textwidth]{B_FDC_5.png}
        \caption{FDC original vs. FDC con 5 bines.}
        \label{fig:B_FDC_5}
    \end{subfigure}
    \hfill
    \begin{subfigure}[b]{0.46\textwidth}
        \includegraphics[width=\textwidth]{B_FDCdiff_5.png}
        \caption{Diferencia absoluta entre FDCs.}
        \label{fig:B_FDCdiff_5}
    \end{subfigure}
    \caption{Evaluación del criterio de refinamiento para el caso de 5 bines.}
    \label{fig:B_FDC_5_5}
\end{figure}

\section*{Caso con 10 bines}
\addcontentsline{toc}{section}{Caso con 10 bines}

En la Figura~\ref{fig:B_letargia_10} se muestra la distribución de letargía aproximada con 10 bines. En este caso, el histograma logra capturar la delta en $1~MeV$ y comienza a mejorar la aproximación de la distribución en la región de termalización y en la región epitérmica. En la Figura~\ref{fig:B_FDC_10} se presentan las FDCs de la distribución original de referencia y la obtenida con 10 bines, mientras que en la Figura~\ref{fig:B_FDCdiff_10} se muestra la diferencia absoluta entre ambas FDCs, que sirve como criterio para determinar la ubicación del nuevo bin en la siguiente iteración.

\begin{figure}[H]
    \centering
    \includegraphics[width=\textwidth]{B_letargia_10.png}
    \caption{Distribución de letargía aproximada con 10 bines. Se observa que el método mejora la aproximación de la distribución en la región de termalización y en la región epitérmica.}
    \label{fig:B_letargia_10}
\end{figure}

\begin{figure}[H]
    \centering
    \begin{subfigure}[b]{0.46\textwidth}
        \includegraphics[width=\textwidth]{B_FDC_10.png}
        \caption{FDC original vs. FDC con 10 bines.}
        \label{fig:B_FDC_10}
    \end{subfigure}
    \hfill
    \begin{subfigure}[b]{0.46\textwidth}
        \includegraphics[width=\textwidth]{B_FDCdiff_10.png}
        \caption{Diferencia absoluta entre FDCs.}
        \label{fig:B_FDCdiff_10}
    \end{subfigure}
    \caption{Evaluación del criterio de refinamiento para el caso de 10 bines.}
    \label{fig:B_FDC_10_10}
\end{figure}

\section*{Caso con 25 bines}
\addcontentsline{toc}{section}{Caso con 25 bines}

En la Figura~\ref{fig:B_letargia_25} se muestra la distribución de letargía aproximada con 25 bines. En este caso, el histograma logra capturar la delta en $1~MeV$ y mejora significativamente la aproximación de la distribución en las regiones epitérmica y térmica, salvo para letargías mayores a \~ 22. En la Figura~\ref{fig:B_FDC_25} se presentan las FDCs de la distribución original de referencia y la obtenida con 25 bines, mientras que en la Figura~\ref{fig:B_FDCdiff_25} se muestra la diferencia absoluta entre ambas FDCs, que sirve como criterio para determinar la ubicación del nuevo bin en la siguiente iteración.

\begin{figure}[H]
    \centering
    \includegraphics[width=\textwidth]{B_letargia_25.png}
    \caption{Distribución de letargía aproximada con 25 bines. Se observa que el método logra capturar la distribución de letargía, salvo en la región de letargía mayor a \~ 22, debido a la poca intensidad que tiene la distribución en esa región.}
    \label{fig:B_letargia_25}
\end{figure}

\begin{figure}[H]
    \centering
    \begin{subfigure}[b]{0.46\textwidth}
        \includegraphics[width=\textwidth]{B_FDC_25.png}
        \caption{FDC original vs. FDC con 25 bines.}
        \label{fig:B_FDC_25}
    \end{subfigure}
    \hfill
    \begin{subfigure}[b]{0.46\textwidth}
        \includegraphics[width=\textwidth]{B_FDCdiff_25.png}
        \caption{Diferencia absoluta entre FDCs.}
        \label{fig:B_FDCdiff_25}
    \end{subfigure}
    \caption{Evaluación del criterio de refinamiento para el caso de 25 bines.}
    \label{fig:B_FDC_25_25}
\end{figure}

\section*{Caso con 100 bines}
\addcontentsline{toc}{section}{Caso con 100 bines}

En la Figura~\ref{fig:B_letargia_100} se muestra la distribución de letargía aproximada con 100 bines. En este caso, el histograma logra capturar con precisión la distribución de letargía, salvo en la región de letargía mayor a \~ 22, debido a la poca intensidad que tiene la distribución en esa región. Sin embargo se observa una mejoría con respecto al caso de 25 bines. En la Figura~\ref{fig:B_FDC_100} se presentan las FDCs de la distribución original de referencia y la obtenida con 100 bines, mientras que en la Figura~\ref{fig:B_FDCdiff_100} se muestra la diferencia absoluta entre ambas FDCs, que sirve como criterio para determinar la ubicación del nuevo bin en la siguiente iteración.


\begin{figure}[H]
    \centering
    \includegraphics[width=\textwidth]{B_letargia_100.png}
    \caption{Distribución de letargía aproximada con 100 bines. Se observa que el método logra capturar con precisión la distribución de letargía, salvo en la región de letargía mayor a \~ 22, debido a la poca intensidad que tiene la distribución en esa región.}
    \label{fig:B_letargia_100}
\end{figure}

\begin{figure}[H]
    \centering
    \begin{subfigure}[b]{0.46\textwidth}
        \includegraphics[width=\textwidth]{B_FDC_100.png}
        \caption{FDC original vs. FDC con 100 bines.}
        \label{fig:B_FDC_100}
    \end{subfigure}
    \hfill
    \begin{subfigure}[b]{0.46\textwidth}
        \includegraphics[width=\textwidth]{B_FDCdiff_100.png}
        \caption{Diferencia absoluta entre FDCs.}
        \label{fig:B_FDCdiff_100}
    \end{subfigure}
    \caption{Evaluación del criterio de refinamiento para el caso de 100 bines.}
    \label{fig:B_FDC_100_100}
\end{figure}







\begin{biblio}
\bibliography{mibib}
\end{biblio}


\begin{postliminary}

% \begin{seccion}{Publicaciones asociadas}
%   \begin{enumerate}
%   \item Mi primer aviso en la revista \textbf{ABC}, 1996
%   \item Mi segunda publicaci\'{o}n en la revista \textbf{ABC}, 1997
%   \end{enumerate}
% \end{seccion}

\begin{seccion}{Agradecimientos}
A todos los que se lo merecen, por merecerlo.
\end{seccion}

\end{postliminary}

\end{document}

