% -----------------------------------------------------------------
% Desenvolvido por Filipe Fernandes para o LIPES (Laboratório de Inovação, Pesquisa e Engenharia de Software)
% filipe.fernandes@ifsudestemg.edu.br
% ----------------------------------------------------------------
% Adaptado de Template Latex - Apresentação - UFMG
% https://www.overleaf.com/latex/templates/template-latex-apresentacao-ufmg/ygvvbrrsqgkv
% -----------------------------------------------------------------
% Licença Creative Commons CC BY 4.0
% -----------------------------------------------------------------
% PARA CORRER pdflatex -shell-escape main.tex
\documentclass[aspectratio=169,english]{beamer}

% a opção hideSubsectionTitle esconde o título das subseções
\usepackage{templateLIPES}      % o arquivo templateLIPES.sty possui todo o estilo de formatação
\usepackage[portuguese]{babel}     % mude o idioma, caso necessite
\usepackage[alf, abnt-emphasize=bf, abnt-etal-text=it]{abntex2cite}

\setbeamertemplate{bibliography item}{}
\renewcommand{\theenumiv}{}

\begin{document}

\titulo{Título da Apresentação}
\subtitulo{Subtítulo}       % caso não haja, comente

\autor{Nome do autor}
\orientador{Prof./Profa. Me./Dr./Dra. Nome Completo}        % caso não haja, comente
\coorientador{Prof./Profa. Me./Dr./Dra. Nome Completo}      % caso não haja, comente

\curso{Bacharelado em Sistemas de Informação}

\local{Manhuaçu}
\dia{4}
\mes{fevereiro}
\ano{2025}

% NÃO REMOVA!
\begin{frame}[plain]
    
    \begin{tikzpicture}[overlay,remember picture]
        \node[left=-0.15cm] at (current page.0){
            \includegraphics[scale=0.145]{imagens/capaLIPES}
        };
    \end{tikzpicture}

    \titlepage
    
\end{frame}

\section[Resumen]{}

\begin{frame}[allowframebreaks]
    \frametitle{Resumen}
    \tableofcontents
\end{frame}

% CONTEÚDO -----------------------------------------------------------------

\section{Introdução}
\begin{frame}{Introdução}
    \begin{itemize}
        \item Teste 1
        \item Teste 2
        \item Teste 3
    \end{itemize}
\end{frame}

\section{Figura}
\begin{frame}{Figura}
    \begin{figure}
        \centering
        \includegraphics[width=0.6\linewidth]{imagens/1_seminario_LIPES.jpeg}
        \caption{1º seminário do LIPES em 27/01/2025}
        \label{fig:1semlipes}
    \end{figure}
\end{frame}

\section{Código}
\begin{frame}[fragile]{Código}
\begin{minted}{python}
from qiskit import QuantumRegister, QuantumCircuit
from numpy import pi

qreg_q = QuantumRegister(2, 'q')
circuit = QuantumCircuit(qreg_q)

circuit.h(qreg_q[0])
circuit.cx(qreg_q[0], qreg_q[1])
\end{minted}
\end{frame}

\section{Citação}
% \begin{frame}{Citação}
%     \begin{itemize}
%         \item De acordo com \citeonline{fernandes2017} ...
%         \item ... lorem ipsum \cite{fernandes2023}.
%     \end{itemize}
% \end{frame}

% FIM DO CONTEÚDO -----------------------------------------------------------------

% NÃO REMOVA!
\section{Referências}
\begin{frame}[allowframebreaks]
    \addtocounter{framenumber}{-1}
    \frametitle{Referências}
    \scriptsize
    % \bibliographystyle{abntex2-alf-en}  % mude para qualquer arquivo .bst
    \bibliography{referencias}
\end{frame}     
\end{document}